\section{Policy regarding academic performance}\footnote{Physics grad
  handbook XIII}

When a student is admitted into the CMSE graduate program it is with
the full expectation that they will thrive academically as scholars
and developing scientists. However, sometimes a student's academic
performance does not meet the expectations of the students and our
faculty.  This section deals with problems and standards for academic performance.

A 3.0 cumulative grade point average (GPA) is the minimum University
standard. Research credits are not considered in determining the
GPA. Attainment of the minimum GPA is, however, an insufficient
indicator of potential for success in other aspects of the program and
in the research field. The student's Guidance committee is responsible
for evaluation of the student's research competency and their rate of
progress toward their degree. 

The accumulation of grades below 3.0 in more than three courses of
three or more credits or ``deferred'' in more than three courses of
three or more credits at any given time, or a combination of the above
in excess of four courses automatically removes the student from
candidacy for the degree. Until the first official Guidance Committee
report is filed, all courses on the student's record are considered
part of the required program.

To remain in good standing the student also needs to follow
Departmental as well as University rules for completing their degree
requirements in a timely manner.  If a student is not making
\textbf{timely and reasonable progress} towards his/her degree in
terms of completing coursework or taking necessary exams, within 30
days following their annual meeting with the Director of the Graduate
Studies, the student should receive a letter from the Department Chair
specifying deficiencies and describing the exact steps, with a time
table, to get back to good standing.  There will be a space on this
letter for the student to respond in writing if they disagree either
with the deficiencies listed or with the steps and time table for
remediation. These responses will be a part of the student's file.

It is a disservice to permit a student to continue toward the advanced
degree without the necessary qualifications for retention, including a
high level of motivation, commitment, and aptitude.  Judgment
regarding retention is made by the student's graduate advisor and/or
the Guidance Committee, in consultation with the Graduate Program
Director and if needed the Department Chairperson.  If a majority of
the Guidance Committee decides that a
student lacks such standards, he/she may be asked to withdraw
according to the procedures as defined in the Graduate Student Rights
and Responsibilities document which can be obtained at
\url{www.msu.edu/students/Splife/gradrights.html}.

The student has a right to receive a warning when academic performance
is judged to be unsatisfactory (see the document Graduate Students
Rights and Responsibilities (GSSR) section 2.4.8.1 and 2.4.8.2).  The
student has a right to access their educational records including the
academic file the department keeps on them (GSSR 3.2.3).  Request to
view and/or copy the file should be made through the department
Graduate Secretary.

If the student does not satisfy the Qualifying exam requirement (see
Section~\ref{sec:qual_exam}) he/she will not be allowed to proceed
towards the M.S./Ph.D. degree.  If after successfully completing the
Qualifying exam the student fails to pass in the Ph.D. Comprehensive
examination (see Section~\ref{sec:comp_exam}), he/she will be
dismissed from the program.  The Oral examinations for the Master's
and Ph.D. degrees are pass/fail.  A student who fails the Master's
Dissertation Defense or Ph.D. Dissertation Defense will be given one
opportunity to repeat the examination within six months (to be decided
by the Guidance Committee).  If the student fails the exam a second
time, he/she will be dismissed from the program.


Further information on rights and responsibilities of graduate
students can be found at the website of the Office of the Ombudsman,
\url{http://www.msu.edu/unit/ombud/}.

 
