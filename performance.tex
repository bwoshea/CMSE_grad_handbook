\section[Policy regarding academic performance]{Policy regarding
  academic performance\footnote{Adapted with permission from the MSU
    Department of Physics \& Astronomy
    \href{https://www.pa.msu.edu/grad/GradHandbook_Aug2015.pdf}{Handbook
      for Graduate
      Students}, Section XIII}}

When a student is admitted into the CMSE graduate program, it is with
the full expectation that they will thrive academically as scholars
and developing scientists. However, sometimes a student's academic
performance does not meet the expectations of the student and our
faculty.  This section deals with problems and standards for academic
performance.

\subsection{Grades}

\noindent \textbf{College Regulations:} A 3.0 cumulative grade point
average (GPA) is a 3.0 average on courses in the student's Program
Plan.  (Note that courses that have already been taken cannot be
dropped from the Program Plan.)  Research credits are not considered
in determining the GPA for either the University or College standard,
and courses not in the Program Plan do not count toward the College
standard.

\noindent \textbf{Department Regulations:} The accumulation of grades
below 3.0 in more than three courses of three or more credits or
``deferred'' in more than three courses of three or more credits at
any given time, or a combination of the above in excess of four
courses automatically removes the student from candidacy for the
degree.  Students will be notified of this in writing via email or
paper mail. Until the first official Guidance Committee report is
filed, all courses on the student's record are considered part of the
required program in their Program Plan


% A 3.0 cumulative grade point average (GPA) is the minimum University
% standard.  The College of Engineering policy on grades, which is more
% restrictive, is a 3.0 average on courses in the student's Program
% Plan.  (Note that courses that have already been taken cannot be
% dropped from the Program Plan.)  Research credits are not considered in determining the
% GPA for either the University or College standard, and courses not in
% the Program Plan do not count toward the College standard. 

% The accumulation of grades below 3.0 in more than three courses of
% three or more credits or ``deferred'' in more than three courses of
% three or more credits at any given time, or a combination of the above
% in excess of four courses automatically removes the student from
% candidacy for the degree.  Students will be notified of this in
% writing via email or paper mail.  Until the first official Guidance Committee
% report is filed, all courses on the student's record are considered
% part of the required program in their Program Plan.

\subsection{Progress toward degree}

Attainment of the minimum GPA is, however, an insufficient
indicator of potential for success in other aspects of the program and
in the research field. The student's Guidance committee is responsible
for evaluation of the student's research competency and their rate of
progress toward their degree. 

To remain in good standing, the student also needs to follow
Departmental as well as University rules for completing their degree
requirements in a timely manner, and a student must continuously be
making \textbf{satisfactory progress} towards his/her degree.  The
Department's criteria for satisfactory academic progress includes:
course credits completed per semester, the nature of these courses,
the grades received, successful completion of required
subject/qualifying/comprehensive examinations, and progress in
completing M.S. or Ph.D. dissertation research.  A student making
satisfactory progress will be on track to complete their M.S. in no more than two years of full-time study,
and a Ph.D. in approximately six years of full-time study beyond the
Bachelor's degree.  In addition to
satisfactory progress toward completing the degree, continuation of
graduate support would depend upon the following: the recipient has
performed the assigned duties satisfactorily; past level of support
and total number of semesters of support; the availability of funds to
continue the current level of financial assistance; the needs of the
Department for the particular services for which the recipient is
qualified to perform.  When resources for financial aid are limited
and the demand of aid exceeds the amount of funds available,
continuation of financial aid for an individual will depend upon merit
relative to others requesting aid and the needs of the Department to
fulfill its overall mission of teaching, research, and outreach.  
 
If a student is not making \textbf{satisfactory progress} towards
his/her degree as defined above, within 30 days following the
documentation of the deficiency, the student will receive a letter
from the Graduate Director specifying deficiencies and describing the
exact steps, with a time table, to get back to good standing.  There
will be a space on this letter for the student to respond in writing
(within no more than two weeks after receipt of the letter)
if they disagree either with the deficiencies listed or with the steps
and time table for remediation. These responses will be a part of the
student's file.

It is a disservice to permit a student to continue toward the advanced
degree without the necessary qualifications for retention, including a
high level of motivation, commitment, and aptitude.  Judgment
regarding retention is made by the student's graduate advisor and/or
the Guidance Committee, in consultation with the Graduate Program
Director and if needed the Department Chairperson.  If, based on the
annual evaluation, the majority of the Guidance Committee decides that a
student lacks such standards, has been notified of deficiencies, and
has not rectified these deficiencies within a reasonable period of
time,
 he/she may be asked to withdraw
according to the procedures as defined in the Graduate Student Rights
and Responsibilities document, which can be obtained at
\url{www.msu.edu/students/Splife/gradrights.html}.

\subsection{Student rights}

The student has a right to receive a written warning when academic
performance is judged to be unsatisfactory (see the document Graduate
Students Rights and Responsibilities (GSSR) sections 2.4.8.1 and
2.4.8.2).  The student has a right to access their educational records
including the academic file the department keeps on them (GSSR 3.2.3),
but excluding any confidential materials (such as the letters of
recommendation submitted on the student's behalf if the student waived
the right to see these letters).  Request to view and/or copy the file
should be made in writing through the department Graduate Secretary.

\subsection{Qualifying and Comprehensive Examinations}

If the student does not satisfy the Qualifying exam requirement (see
Section~\ref{sec:qual_exam}), they will not be allowed to proceed
towards the M.S./Ph.D. degree.  If after successfully completing the
Qualifying exam the student fails to pass in the Ph.D. Comprehensive
examination (see Section~\ref{sec:comp_exam}), he/she will be
dismissed from the program.  

Note that it is possible to appeal the results of the Qualifying or
Comprehensive examinations by written request to the CMSE Graduate
Studies Committee. Petition requests will only be granted for students
whose level of research progress, as reported by their dissertation
advisor, makes it clear that they are highly likely to successfully
complete the doctoral program with an acceptable level of research
productivity and in a reasonable amount of time.  Please consult
Section~\ref{sec:appeals} for further information about the appeals
process.

\subsection{Ph.D. oral examination}

The Oral examination for the Ph.D. degree is pass/fail.  A student who fails the Ph.D. Dissertation Defense will be given one
opportunity to repeat the examination after an
adequate amount of time to preprae has been given -- a period of time
that must be at least one month, but no more than six months.  If the student fails the exam a second
time, he/she will be dismissed from the program.

\subsection{Further information}

Further information on rights and responsibilities of graduate
students can be found at the website of the Office of the Ombudsman,
\url{https://ombud.msu.edu/}.

 
