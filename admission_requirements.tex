\section{Admission requirements for MS and PhD programs}
\label{sec:grad_admission}

Students who wish to apply to either the master's or doctoral program in Computational
Mathematics, Science and Engineering must, at the time of entering
the program, have a Bachelor's degree in mathematics, statistics,
computer science, or any science or engineering field.  Beyond this
requirement, a student must have:

\begin{itemize}
\item Coursework in calculus through differential equations.

\item A course in basic linear algebra or equivalent training through
  a related course.

\item Competency in basic statistics through an introductory-level
  course or equivalent practical
  experience

\item Ability to program competently in at least one
  compiled programming language that is commonly-used in scientific
  computing and data science
  (e.g., Python, C/C++, etc.)

\end{itemize}

The competitive applicant to the PhD program will also have some experience outside of
their coursework in scientific computing or data science that
demonstrates their aptitude for success in a PhD program.  This
experience could include, but is not limited to:  

\begin{itemize}
\item Undergraduate research experience with a strong computational
  component, possibly in a Research Experience for Undergraduates program or working directly with an
  individual faculty member at their home institution.

\item An internship at a national laboratory or a company, where the
  student is working on a
  project that has a significant computational modeling and/or data
  science focus.

\item Independent contributions to open-source programs or libraries,
  particularly with a computational modeling and/or data science
  focus.
\end{itemize}

Applications to the M.S. or PhD program in Computational Mathematics, Science and
Engineering must include:

\begin{itemize}
\item Official transcripts for all prior undergraduate or graduate
  degrees and coursework
\item A resume or CV
\item An academic statement \textbf{of no more than two pages in
    length} that follows the
  \href{http://www.egr.msu.edu/academics/graduate/academic-personal-statements-guidelines/#academic}{academic
    statement
    guidelines from the College of Engineering}, 
  describing:  

\begin{itemize}
  \item The applicant's prior programming, computational, and research experience;  
  \item The applicant's goals in pursuing a PhD in CMSE;  
  \item  The CMSE faculty with whom an applicant may be interested in
    pursuing their dissertation research.  
\end{itemize}

\item A personal statement \textbf{of no more than two pages in
    length} that follows the
  \href{http://www.egr.msu.edu/academics/graduate/academic-personal-statements-guidelines/#personal}{personal
    statement
    guidelines from the College of Engineering}, explaining
why you are motivated to pursue
  a graduate degree in Computational Mathematics, Science and Engineering

\item At least three letters of recommendation that address the
  applicant's past accomplishments and potential for success in
  pursuing independent research in computational and/or data science.

\item The general Graduate Record Examination (GRE)

\item \textbf{International students have additional requirements},
  including (but not limited to) official English proficiency test
  scores from either the Test of English as a Foreign Language
  (TOEFL), IELTS, or PTE-A, with an appropriate minimum score. Information concerning 
  Furthermore, MSU requires all incoming \textbf{\textit{admitted}}
  students pursuing degrees or who have earned degrees from
  universities in China to submit a certification report (English
  version) through the China Academic Degrees and Graduate Education
  Development Center (CDGDC) for their final bachelor degree
  transcripts and bachelor degree.  Please consult
  \href{https://grad.msu.edu/internationalapplicants}{this page} and
  the CMSE PhD program website for additional information.


\end{itemize}


\noindent
{\large \textbf{New Service Provided by the English Language Center}} 
\\

The ELC conducts applicant screening interviews via Skype and Zoom
jointly with senior representatives from participating departments.
The aim of the interview is to determine the likelihood of the
applicant passing the MSU Speaking Test upon arrival on campus.
Departments may obtain more information and request this service by
sending an email to \href{mailto:testing@elc.msu.edu}{testing@elc.msu.edu}.

\begin{itemize}

\item If international students are admitted on a provisional basis
  because of language proficiency requirements, they can be issued an
  I-20 for language studies only.   This I-20 is limited to a maximum
  of 2 years.  The student would need to be tested at the English
  Language Center upon arrival and begin studying in the English level
  determined by that test.  Once the student meets the departmental
  requirements for language, the student may be issued a
  degree-seeking I-20.   If the student has not met the stated
  language proficiency requirement for department/program admission at
  the end of two years, the student cannot continue to enroll for
  courses. 

\item Provisional admission for international students can be granted
  ONLY for language deficiencies.  Regulations will not allow the
  issuing of an I-20 for provisional admissions for academic reasons.   

\end{itemize}






Please consult the College of Engineering's instructions on
\href{http://www.egr.msu.edu/academics/graduate/how-to-apply}{how to
apply to an Engineering graduate program}, and pay particular
attention to the
\href{http://www.egr.msu.edu/academics/graduate/academic-personal-statements-guidelines/}{guidelines
for academic and personal statements}.  The academic statement is
particularly critical for students who are interested in being
nominated for College- and University-wide fellowships, and the
guidelines described by the linked document will provide the
department with the needed information for making a strong case.

\vspace{3mm}
\noindent \textbf{Note about the M.S. program:} At present, admission
directly to the Master of Science degree is only available in
exceptional circumstances.  Students interested in pursuing a graduate
degree in Computational Mathematics, Science and Engineering must
apply directly to the PhD program.  Please contact the
\href{mailto:cmsegrad@msu.edu}{CMSE Director of Graduate Studies} with
any questions.

