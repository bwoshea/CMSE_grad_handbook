\section{Admission requirements for MS and PhD programs}
\label{sec:grad_admission}

Students who wish to apply to either the master's or doctoral program in Computational
Mathematics, Science and Engineering must, at the time of entering
the program, have a Bachelor's degree in mathematics, statistics,
computer science, or any science or engineering field.  Beyond this
requirement, a student must have:

\begin{itemize}
\item Coursework in calculus through differential equations.

\item A course in basic linear algebra or equivalent training through
  a related course.

\item Competency in basic statistics through an introductory-level
  course or equivalent practical
  experience

\item Ability to program competently in at least one commonly-used programming language
  (e.g., Python, C/C++, Java, etc.)

\end{itemize}

The competitive applicant to the PhD program will also have some experience outside of
their coursework in scientific computing or data science that
demonstrates their aptitude for success in a PhD program.  This
experience could include, but is not limited to:  

\begin{itemize}
\item Undergraduate research experience with a strong computational
  component, possibly in a Research Experience for Undergraduates program or working directly with an
  individual faculty member at their home institution.

\item An internship at a national laboratory or a company, where the
  student is working on a
  project that has a significant computational modeling and/or data
  science focus.

\item Independent contributions to open-source programs or libraries,
  particularly with a computational modeling and/or data science
  focus.
\end{itemize}

Applications to the M.S. or PhD program in Computational Mathematics, Science and
Engineering must include:

\begin{itemize}
\item Official transcripts for all prior undergraduate or graduate degrees and coursework
\item A resume or CV
\item An academic statement \textbf{of no more than two pages in
    length} that follows the
  \href{http://www.egr.msu.edu/academics/graduate/academic-personal-statements-guidelines/#academic}{academic
    statement
    guidelines from the College of Engineering}, 
  describing:  

\begin{itemize}
  \item The applicant's prior programming, computational, and research experience;  
  \item The applicant's goals in pursuing a PhD in CMSE;  
  \item  The CMSE faculty with whom an applicant may be interested in
    pursuing their dissertation research.  
\end{itemize}

\item A personal statement \textbf{of no more than two pages in
    length} that follows the
  \href{http://www.egr.msu.edu/academics/graduate/academic-personal-statements-guidelines/#personal}{personal
    statement
    guidelines from the College of Engineering}, explaining
why you are motivated to pursue
  a graduate degree in Computational Mathematics, Science and Engineering

\item At least three letters of recommendation that address the
  applicant's past accomplishments and potential for success in
  pursuing independent research in computational and/or data science.

\end{itemize}

Please consult the College of Engineering's instructions on
\href{http://www.egr.msu.edu/academics/graduate/how-to-apply}{how to
apply to an Engineering graduate program}, and pay particular
attention to the
\href{http://www.egr.msu.edu/academics/graduate/academic-personal-statements-guidelines/}{guidelines
for academic and personal statements}.  The academic statement is
particularly critical for students who are interested in being
nominated for College- and University-wide fellowships, and the
guidelines described by the linked document will provide the
department with the needed information for making a strong case.

\noindent \textbf{Note about the M.S. program:} At present, admission
directly to the Master of Science degree is only available in
exceptional circumstances.  Students interested in pursuing a graduate
degree in Computational Mathematics, Science and Engineering must
apply directly to the PhD program.  Please contact the
\href{mailto:cmsegrad@msu.edu}{CMSE Director of Graduate Studies} with
any questions.
