\section{Graduate Certificate in High-Performance Computing}
\label{sec:cert_hpc}

\subsection{Certificate description}

The Graduate Certificate in High Performance Computing is intended for
graduate students in any discipline who have significant prior
computational experience.  The purpose of this certificate is to
completment students' degree programs with a set of courses that
provide students with a broad exposure to parallel computing
methodology, and give them experience with computational and data
science challenges that require parallel and/or high-performance
computing in order to solve effectively.

Students that have completed this certificate will be able to:

\begin{itemize}
\item  Demonstrate a high-level understanding of functional and
  object-oriented computer programming as applied to a range of
  problems in computational and data science.

\item  Analyze problems in terms of the algorithms and pre-existing
  computational tools required to solve a range of problems in
  computational and data science, and write a program to efficiently
  solve the problem on modern parallel computers and specialized
  hardware (e.g., graphics processing units).

\item  Construct and implement models of a variety of systems using
  modern parallel programming techniques and software development
  techniques, and use these models/simulations to gain understanding
  of these systems.

\item  Apply some subset of discipline-focused or methodology-focused
topics in computational and data science to solve problems in the
student’s primary discipline.

\end{itemize}

\subsection{Certificate requirements}

The proposed Graduate Certificate in High Performance Computing
consists of at least three courses comprising a minimum of 9 credit
hours, taken from the two categories listed below. The targets of the
certificate program are graduate students in any discipline with
interest in applying computational and data science approaches that
require parallel and/or high-performance computing to their research
problems, or who generally desire an education in parallel
computational methodology. To facilitate this goal, in addition to
there being no restriction on graduate student discipline, students
can apply for the certificate at any time prior to receiving their
degree (either Master’s or PhD), and can apply for the certificate
after taking all the necessary courses.

Note that credit from courses whose focus is largely or primarily an
introduction to programming and/or basic numerical methods (i.e., CMSE
801, CMSE 802, CSE 801, or other comparable courses) \textbf{will not count
for credit} toward this certificate. In addition, 400-level
computational coursework may not count for credit toward this
certificate without the permission of the CMSE graduate certificate
advisor. The primary circumstance where a 400-level course may be
acceptable for credit toward this certificate program is when an
equivalent 800-level course is unavailable (e.g., a highly specialized
400-level combined undergraduate and graduate course.) Students that
have questions about any particular course are strongly encouraged to
consult the CMSE graduate certificate advisor.

Categories of courses:

\begin{enumerate}

\item CMSE/CSE-822, Parallel Computing (3 credits)  

\item Two or more additional courses, which may include further CMSE
  courses at the 800 level or above, courses from the list of non-CMSE
  courses in Section~\ref{sec:courses}, or any other 800-
  or 900-level computational science or data science-focused courses
  as approved by the CMSE graduate advisor (6 or more credits).

\end{enumerate}

\subsection{Admission and graduation requirements}

Graduate students enrolled at Michigan State University in any
discipline or college may pursue this graduate certificate.
Furthermore, students can apply for the certificate at any time prior
to receiving their primary degree (either Master’s or PhD), and can
apply for the certificate after taking all the necessary courses.

Graduate students pursuing either the Master of Science in CMSE, the
Doctor of Philosophy in CMSE, or a dual PhD in CMSE and a second
subject \textbf{may not also receive} a graduate certificate in
High Performance Computing.

In order to obtain this graduate certificate the student must have at
least a 3.0 average in the courses that are applied to the
certificate.

