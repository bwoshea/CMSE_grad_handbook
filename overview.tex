\section{Overview}
\label{sec:overview}

The use of computational methods and tools to solve important problems
is a fact of life in the 21st century in the sciences, engineering,
and many other fields.  Knowledge of how these methods work, and how
to use them effectively, has become crucial to the success of students
entering these fields.

To that end, Michigan State University's \href{http://cmse.msu.edu}{Department of Computational
Mathematics, Science and Engineering} (CMSE) has developed several
curricula at the graduate level that are intended to provide MSU
students with the computational skills that they need to thrive in the
21st century workforce.  Broadly speaking, this includes:

\begin{itemize}
\item The ability to visualize and explore large quantities of data to
  find important relations and trends (i.e., `data science' and/or
  `big data').

\item The ability to formulate and implement  software models to
  explain and explore a wide variety of systems and phenomena.

\item The ability to effectively use modern computational hardware,
  including cloud computing, massively parallel supercomputers, and
  hardware accelerators.

\end{itemize}

MSU students have a wide variety of needs with regards to
computational and data science, which may range from taking a single
course on computational modeling or numerical methods through
completing a PhD in scientific computing or data science.  This
document provides details about the courses and curricula available
through the Department of Computational Mathematics, Science and
Engineering, as well as policies and procedures pertaining to graduate
students enrolled in degree programs in the Department.

Please note that this document is meant to be consistent with
University and College of Engineering policies, as linked to in
Section~\ref{sec:resources_policy} of this Handbook.  In the event
where policies outlined in this Handbook are inconsistent with those
outlined in the documents in Section~\ref{sec:resources_policy}, the
University-level and College-level policies in those documents
override this handbook.
