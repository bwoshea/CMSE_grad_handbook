\section{Overview}
\label{sec:overview}


The Department of Computational Mathematics, Science and Engineering
(CMSE) is unique among computational academic units nationally; it is
the first to comprehensively treat computation as the ``triple
junction'' of algorithm development and analysis, high performance
computing, and applications to scientific and engineering modeling and
data science. This approach recognizes computation as a new discipline
rather than being decentralized into isolated sub-disciplines. CMSE,
jointly administered by Michigan State University's Colleges of
Natural Science and Engineering, will enable application-driven
computational modeling (``pull''), while also exposing disciplinary
computational scientists to advanced tools and techniques (“push”),
which will ignite new transformational connections in research and
education.  The Department of CMSE is focused on applications that are
aligned with experimental expertise at Michigan State University. CMSE
is targeting applications in the physical, biological and engineering
sciences. CMSE faculty members focus on the science of algorithm
development, as key methods developed in this area bridge many
application core areas in science.

\vspace{2mm}
\noindent
\textbf{Scientific Computing,} also referred to as ``computational science,''
focuses on the development of predictive computer models of the world
around us. As study of physical phenomena through experimentation has
become impossible, impractical and/or expensive, computational
modeling has become the primary tool for understanding—equal in
stature to analysis and experiment. Although we can now design an
entire commercial aircraft through simulation alone (e.g., the Boeing
777), there are many fundamental problems in science and engineering
that are beyond the scope of modern computers with current
computational methods. The discipline of scientific computing is the
development of new methods that make challenging problems tractable on
modern computing platforms, providing scientists and engineers with
key windows into the world around us.

\vspace{2mm}
\noindent
\textbf{Data Science} focuses on the development of
tools designed to find trends within datasets that help scientists who
are challenged with massive amounts of data to assess key relations
within those datasets. These key relations provide hooks that allow
scientists to identify models which, in turn, facilitate making
accurate predictions in complex systems. For example, a key data
science goal on the biological side would be better care for patients
(e.g., personalized medicine). Given a patient’s genetic makeup, the
proper data-driven model would identify the most effective treatment
for that patient.

\vspace{2mm}
Knowledge of the methods of scientific computing and data science is
crucial to the solution of cutting-edge problems in the sciences,
engineering, and many other fields.  To that end, Michigan State
University's \href{http://cmse.msu.edu}{Department of Computational
Mathematics, Science and Engineering} (CMSE) has developed several
curricula at the graduate level that are intended to provide MSU
students with the computational skills that they need to thrive in the
21st century workforce.  Broadly speaking, this includes:

\begin{itemize}
\item The ability to visualize and explore large quantities of data to
  find important relations and trends (i.e., `data science' and/or
  `big data').

\item The ability to formulate and implement  software models to
  explain and explore a wide variety of systems and phenomena.

\item The ability to effectively use modern computational hardware,
  including cloud computing, massively parallel supercomputers, and
  hardware accelerators.

\end{itemize}

MSU students have a wide variety of needs with regards to
computational and data science, which may range from taking a single
course on computational modeling or numerical methods through
completing a PhD in scientific computing or data science.  This
document provides details about the courses and curricula available
through the Department of Computational Mathematics, Science and
Engineering, as well as policies and procedures pertaining to graduate
students enrolled in degree programs in the Department.

Please note that this document is meant to be consistent with
University and College of Engineering policies, as linked to in
Section~\ref{sec:resources_policy} of this Handbook.  In the event
where policies outlined in this Handbook are inconsistent with those
outlined in the documents in Section~\ref{sec:resources_policy}, the
University-level and College-level policies in those documents
override this handbook.
