\section{Academic Integrity Policy}\footnote{ECE graduate handbook section 8}

\subsection{The MSU perspective}

Each graduate student shall have the document Guidelines for Integrity
in Research and Creative Ideas. See \red{Section NNN} for access to
this document. The conduct of research and creative activities by
faculty, staff, and students is central to the mission of Michigan
State University and is an institutional priority. Faculty, staff, and
students work in a rich and competitive environment for the common
purpose of learning, creating new knowledge, and disseminating
information and ideas for the benefit of their peers and the general
public. The stature and reputation of MSU as a research university are
based on the commitment of its faculty, staff, and students to
excellence in scholarly and creative activities and to the highest
standards of professional integrity.  

As a partner in scholarly endeavors, MSU is committed to creating an
environment that promotes ethical conduct and integrity in research
and creative activities. Innovative ideas and advances in research and
creative activities have the potential to generate professional and
public recognition and, in some instances, commercial interest and
financial gain. In rare cases, such benefits may become motivating
factors to violate professional ethics. Pressures to publish, to
obtain research grants, or to complete academic requirements may also
lead to an erosion of professional integrity.  

Breaches in professional ethics range from questionable research
practices to misconduct. The primary responsibility for adhering to
professional standards lies with the individual scholar. It is,
however, also the responsibility of advisors and of the disciplinary
community at large. Passive acceptance of improper practices lowers
inhibitions to violate professional ethics. 

Integrity in research and creative activities is based not only on
sound disciplinary practice but also on a commitment to basic personal
values such as fairness, equity, honesty, and respect. These
guidelines are intended to promote high professional standards by
everyone — faculty, staff, and students alike.
  
For further information and training, graduate students are \textbf{required}
to participate in the Responsible Conduct of Research workshop series,
sponsored by the Office of the Vice President for Research and
Graduate Studies and by the Graduate Dean. Information on this series
is available at the graduate school web site:
\url{http://www.msu.edu/user/gradschl/}

\subsubsection{Key principles}

Integrity in research and creative activities embodies a range of practices that includes: 

\begin{itemize}
\item Honesty in proposing, performing, and reporting research. 

\item Recognition of prior work. 

\item Confidentiality in peer review. 

\item Disclosure of potential conflicts of interest. 

\item Compliance with institutional and sponsor requirements. 

\item Protection of human subjects and humane care of animals in the
  conduct of research. 

\item Collegiality in scholarly interactions and sharing.  

\item Adherence to fair and open relationships between senior
scholars and their co-workers. 

\end{itemize}

\textbf{Honesty in proposing, performing, and reporting research:} The
foundation underlying all research is uncompromising honesty in
presenting one’s own ideas in research proposals, in performing one’s
research, and in reporting one’s data. Detailed and accurate records
of primary data must be kept as unalterable documentation of one’s
research and must be available for scrutiny and critique. It is
expected that researchers will always be truthful and explicit in
disclosing what was done, how it was done, and what results were
obtained. To this end, research aims, methods, and outcomes must be
described in sufficient detail such that others can judge the quality
of what is reported and can reproduce the data. Results from valid
observations and tests that run counter to expectations must be
reported along with supportive data.
 
\textbf{Recognition of prior work:} Research proposals, original
research, and creative endeavors often build on one’s own work and
also on the work of others. Both published and unpublished work must
always be properly credited. Reporting the work of others as if it
were one’s own is plagiarism. Graduate advisors and members of
guidance committees have a unique role in guiding the independent
research and creative activities of students. Information learned
through private discussions or committee meetings should be respected
as proprietary and accorded the same protection granted to information
obtained in any peer review process. 

\textbf{Confidentiality in peer review:} Critical and impartial review
by respected disciplinary peers is the foundation for important
decisions in the evaluation of internal and external funding requests,
allocation of resources, publication of research results, granting of
awards, and in other scholarly decisions. The peer-review process
involves the sharing of information for scholarly assessment on behalf
of the larger disciplinary community. The integrity of this process
depends on confidentiality until the information is released to the
public. Therefore, the contents of research proposals, of manuscripts
submitted for publication, and of other scholarly documents under
review should be considered privileged information not to be shared
with others, including students and staff, without explicit permission
by the authority requesting the review. Ideas and results learned
through the peer-review process should not be made use of prior to
their presentation in a public forum or their release through
publication. 

\textbf{Disclosure of potential conflicts of interest:} There is real
or perceived conflict of interest when a researcher has material or
personal interest that could compromise the integrity of the
scholarship. It is, therefore, imperative that potential conflicts of
interest be considered and acted upon appropriately by the
researcher. Some federal sponsors require the University to implement
formal conflict of interest policies. It is the responsibility of all
researchers to be aware of and comply with such requirements. 

\textbf{Compliance with institutional and sponsor requirements:}
Investigators are granted broad freedoms in making decisions
concerning their research. These decisions are, however, still guided,
and in some cases limited, by the laws, regulations, and procedures
that have been established by the University and sponsors of research
to protect the integrity of the research process and the uses of the
information developed for the common good. Although the legal
agreement underlying the funding of a sponsored project is a matter
between the sponsor and the University, the primary responsibility for
management of a sponsored project rests with the principal
investigator and his or her academic unit. 

\textbf{Protection of human subjects and humane care of animals in the
  conduct of research:} Research techniques should not violate
established professional ethics or federal and state requirements
pertaining to the health, safety, privacy, and protection of human
beings, or to the welfare of animal subjects. Whereas it is the
responsibility of faculty to assist students and staff in complying
with such requirements, it is the responsibility of all researchers to
be aware of and to comply with such requirements. 

\textbf{Collegiality in scholarly interactions and sharing of
  resources:} Collegiality in scholarly interactions, including open
communications and sharing of resources, facilitates progress in
research and creative activities for the good of the community. At the
same time, it has to be understood that scholars who first report
important findings are both recognized for their discovery and
afforded intellectual property rights that permit discretion in the
use and sharing of their discoveries and inventions. Balancing
openness and protecting the intellectual property rights of
individuals and the institution will always be a challenge for the
community. Once the results of research or creative activities have
been published or otherwise communicated to the public, scholars are
expected to share materials and information on methodologies with
their colleagues according to the tradition of their discipline. 

\textbf{Adherence to fair and open relationships between senior
  scholars and their coworkers:} The relationship between senior
scholars and their coworkers should be based on mutual respect, trust,
honesty, fairness in the assignment of effort and credit, open
communications, and accountability. The principles that will be used
to establish authorship and ordering of authors on presentations of
results must be communicated early and clearly to all coworkers. These
principles should be determined objectively according to the standards
of the discipline, with the understanding that such standards may not
be the same as those used to assign credit for contributions to
intellectual property. It is the responsibility of the faculty to
protect the freedom to publish results of research and creative
activities. The University has affirmed the right of its scholars for
first publication except for “exigencies of national defense”. It is
also the responsibility of the faculty to recognize and balance their
dual roles as investigators and advisors in interacting with graduate
students of their group, especially when a student’s efforts do not
contribute directly to the completion of his or her degree
requirements. 

Faculty advisors have a particular responsibility to respect and
protect the intellectual property rights of their advisees. A clear
understanding must be reached during the course of the project on who
will be entitled to continue what part of the overall research program
after the advisee leaves for an independent position. Faculty advisors
should also strive to protect junior scholars from abuses by others
who have gained knowledge of the junior scholar’s results during the
mentoring process, for example, as members of guidance committees. 

\subsection{Misconduct in research and creative activities}

Federal and University policies define misconduct to include
fabrication (making up data and recording or reporting them),
falsification (manipulating research materials, equipment or
processes, or changing or omitting data such that the research is not
accurately represented in the record), and plagiarism (appropriation
of another person’s ideas, processes, results, or words without giving
appropriate credit). Serious or continuing non-compliance with
government regulations pertaining to research may constitute
misconduct as well. University policy also defines retaliation against
whistle blowers as misconduct. Misconduct does not include honest
errors or honest differences of opinion in the interpretation or
judgment of data. 

The University views misconduct to be the most egregious violation of
standards of integrity and as grounds for disciplinary action,
including the termination of employment of faculty and staff,
dismissal of students, and revocation of degrees. It is the
responsibility of faculty, staff, and students alike to understand the
University’s policy on misconduct in research and creative activities,
to report perceived acts of misconduct of which they have direct
knowledge to the University Intellectual Integrity Officer, and to
protect the rights and privacy of individuals making such reports in
good faith. 

\subsection{Research involving human subjects}

The University Committee on Research Involving Human Subjects (UCRIHS)
is an Institutional Review Board (IRB). Federal regulations and
University policy require that all research projects involving human
subjects and materials of human origin be reviewed and approved by an
IRB before initiation. Research is defined as ``a systematic
investigation, including research development, testing and evaluation,
designed to develop or contribute to generalizable knowledge.'' The
``generalizable knowledge'' criteria may include developing
publications/papers, theses/dissertations, making public
presentations, etc. A human subject of research is a) a living
individual from whom an investigator obtains data by interaction or
intervention or b) identifiable private information.  

All research involving human subjects and/or data collected from
living human subjects (including preexisting data) is subject to
UCRIHS review. Instructions for applying for approval are available at
the following web site:  \url{http://www.humanresearch.msu.edu/}.

