\section{Doctor of Philosophy in CMSE}
\label{sec:phd}

\subsection{Program overview}
\label{sec:phd_overview}

The overall goal of the PhD program in Computational Mathematics,
Science and Engineering is to give students broad and deep knowledge
of the fundamental techniques used in computational modeling and data
science, significant exposure to at least one application domain, and
to conduct significant original research in algorithms and/or
applications relating to computational and data science.

\vspace{2mm}
\noindent
Students that have completed this PhD program will be able to:

\begin{itemize}
\item  Analyze problems in terms of the algorithms and pre-existing
  computational tools required to solve a range of problems in
  computational and data science, and write programs to efficiently
  solve the problem using cutting-edge computational hardware.  

\item  Construct and implement models and simulations of physical,
  biological, and social phenomena, and use these models/simulations
  to understand experimental or observational data and make testable predictions.  

\item  Apply discipline-focused or methodology-focused topics in
  computational and data science to solve problems in the student's
  application domain of choice.

\item  Conduct  original research and present it in
  peer-reviewed articles, a written dissertation, and orally in a
  variety of venues.  

%\item   \red{\textbf{OTHER THINGS HERE?}}

\end{itemize}


\subsection{Program requirements}
\label{sec:phd_requirements}

Students pursuing a PhD in CMSE must:

\begin{itemize}
\item  Demonstrate proficiency in the following four areas:
  Mathematical Foundations of Data Science, Numerical Methods for
  Differential Equations, Parallel Programming, and Numerical Linear
  Algebra.  Proficiency is demonstrated by passing the qualifying
  exam, which consists of separate subject exams in each of these
  topics.  Students typically enroll in the four core CMSE courses
  that provide instruction in these subject areas (CMSE 820-823; 12
  credits total).  \textbf{The four subject exams must be passed with an
  average grade of 3.5 and no exam having a grade below 3.0}.  See
Sections~\ref{sec:core_courses} and~\ref{sec:qual_exam} for more information.

\item Complete a minimum of 18 additional credits in coursework
  chosen in consultation with, and approved by, the student's
  dissertation committee.  (Students must take a minimum of 30 credit
  hours of non-research-based coursework beyond the Bachelor's level;
  see Appendix~\ref{sec:prior_coursework} for information pertaining
  to students entering with prior graduate coursework.)
  No more than 6 credits of this coursework can be below the 800 level
  (and all such coursework is allowable only with approval of the
  Graduate Director), and none of this coursework can be below the 400
  level.  See Section~\ref{sec:cognate_course}.

\item Complete a minimum of 24, and maximum of 36, dissertation
  research credits (CMSE 999).

\item Pass a comprehensive examination with both written and oral
  components no later than the end of the third year, and at least six months before the defense of their
  dissertation.  See Section~\ref{sec:comp_exam}.

\item Meet at least annually with their dissertation committee to report on
  progress and receive guidance.  This annual meeting must include a
  written annual report by the committee describing the student's
  progress towards their degree.   See Section~\ref{sec:phd_dissertation}.

\item Complete and orally defend a dissertation based on original
  research in algorithms pertaining to, and/or applications of,
  computational and/or data science.  See Section~\ref{sec:phd_dissertation}.

\item All PhD students must complete Responsible Conduct of Research (RCR)
  Training and submit annual reports as specified by the
  \href{https://www.egr.msu.edu/academics/graduate/rcr}{College of
    Engineering}.  See Appendix~\ref{sec:rcr_requirements} for a
  detailed explanation of RCR requirements and a timeline.

\end{itemize}

Each of these requirements is described in more detail below.

\vspace{3mm}
\subsubsection{Core course requirement}
\label{sec:core_courses}

The purpose of the four core courses (in numerical linear algebra,
numerical differential equations, parallel computing, and the
mathematical foundations of data science) is to provide CMSE PhD
students with a broad and deep understanding of the fundamental
algorithms pertaining to computational and data science.

A student with no deficiencies in their mathematics or programming
background should complete all four core courses during the first year
of their PhD (where ``completion'' assumes passage of the subject exams;
see below); at the latest, all four core courses should be completed
by the end of their 2nd year in the program (i.e., within 24 months of
entering the program).  Students can demonstrate proficiency in
one or more of these core courses, and thus be excused from taking
said course(s), by taking the subject exam and receiving a passing
grade (see the ``Qualifying Examination'' section for more details).
Students who demonstrate proficiency in this way are still required to
meet the minimum course credit requirement for the degree (30 credits
beyond the Bachelor's degree).  (The exception to this rule is
students who enter the CMSE PhD program with prior graduate
coursework; see Appendix~\ref{sec:prior_coursework}.)

\vspace{3mm}
\subsubsection{Qualifying examination}
\label{sec:qual_exam}

The qualifying examination in CMSE is composed of four subject exams,
corresponding to the four core CMSE courses (CMSE 820-823). This
requirement is intended to ensure that all students receiving a CMSE
PhD achieve an acceptable level of expertise in the core numerical
algorithms pertaining to computational and data science.  

Each subject exam will be offered twice per academic year.  All four
exams will be offered immediately prior to the start of the fall
semester, and each subject exam will be offered again as the final
exam of the corresponding core CMSE course (parallel programming and
numerical linear algebra in the fall; numerical differential equations
and the mathematical foundations of data science in the spring).

Students will be given three opportunities to pass each subject exam,
and must pass all four subject exams by the end of their second year
in the PhD program (i.e., within 24 months of entering the program).
Students beginning the PhD program in the fall
term may, but are not required to, take one or more subject exams
immediately upon arrival at MSU.  Assuming no deficiencies, all
students must take the subject exams the first time they are offered
after the beginning of their first semester in the PhD program (i.e.,
as the final exam of the corresponding core course for students
entering in the fall).  Students starting the PhD program in the
spring term will not have an option to take subject exams prior to the
start of the spring semester; rather, they must take the subject exams
for the first time at the end of the semester where each core course
is offered.

The subject exams will cover a set of topics set forth by the
Department, which will be provided to the students no less than two weeks in advance of the
exams. The subject exams will be created and graded by a committee of
CMSE faculty that must include the current or most recent instructor
of that course.  The purpose of the committee is to ensure consistency
in topical coverage and level of difficulty between instructors and
exam offerings.  Subject exams may include a practical (i.e.,
programming) component.  Passage of all four subject exams with a
minimum average exam grade of 3.5 and no exam grade below 3.0
constitutes passage of the qualifying exam requirement for the
department.  Graduate students who receive a grade on the qualifying
exam will get formal feedback within a period of  two weeks on
their performance on the exam.


It is possible that a student has already taken graduate coursework,
or obtained equivalent practical experience, in one or more subjects
that correspond to the four course CMSE graduate courses.  Such a
student may be excused from taking that course by demonstrating
proficiency in the material by passing the appropriate subject exam,
with the requirements for passage being the same as for students who
choose to enroll in the course.  In this case, students must still
meet the course credit requirement of 30 credits past the bachelor's
degree by taking other courses.

\vspace{3mm}
\subsubsection{Cognate course requirement}
\label{sec:cognate_course}

The purpose of the cognate requirement is to give students in-depth
expertise in one or more subject areas that are complementary to the
core curriculum and which pertain to their research interests, to
provide in-depth exposure to one or more areas of numerical
methodology, to develop expertise in an additional area of
computational or data science, and/or to fill gaps in a student's
undergraduate education.

The cognate coursework must be chosen by the student in consultation
with their dissertation committee (or the Director of Graduate
Studies, if the student has not yet formed their dissertation
committee), and must include at least 12 credits of 
coursework addressing a single general topic or subject area.  This
cognate can be in any subject area, including an application area,
mathematics, statistics, or computer science.  A minimum GPA of 3.0 is
expected in the cognate area, and no more than 6 credits can be below
the 800 level (with no credits below the 400 level).  Note that this
credit requirement can be waived for students who enter the CMSE PhD
program with prior graduate coursework; see
Appendix~\ref{sec:prior_coursework}.

\vspace{3mm}
\subsubsection{Comprehensive examination}
\label{sec:comp_exam}

There are several goals of the comprehensive examination in the CMSE
PhD program.  This examination enables students:

\begin{itemize}
\item To demonstrate mastery of the current state-of-the-art in their
  chosen area of specialization, as defined by the current body of
  literature in that field.

\item To demonstrate their ability to communicate scientific goals,
  methods, and results in written and oral form. 

\item To demonstrate the ability to construct a realistic and detailed
  plan of research for the rest of their dissertation (which
  presupposes that the student has already begun their research project).

\end{itemize}

The comprehensive exam is also the first opportunity for students to
receive formal feedback from their dissertation committee.

\vspace{2mm}
\noindent
\textbf{Timing:} Students should plan to complete the comprehensive
examination component of their PhD requirements within two years of
beginning the PhD program, and no later than the end of their third
year in the program.  In addition, this requirement must be completed
no less than six months prior to defending their dissertation. 


\vspace{2mm}
\noindent
\textbf{Examination components:} The comprehensive examination has two components: a written project proposal and an oral presentation.

The written project proposal should be approximately 10-15 pages long, and should include the following elements:

\begin{enumerate}
\item  A literature survey describing the current state of their chosen area of specialization.
\item  A summary of the research that the student has done so far, which should form the beginning of their dissertation research.
\item  A proposal for the rest of their dissertation research, including:

\begin{enumerate}
  \item  The scientific motivation and significance of this work.
  \item  The specific aims of the project(s) that will be undertaken.
  \item  A timeline for completion of this work.
\end{enumerate}

\item  A brief description of their post-PhD career goals and a
  description of the types of professional development opportunities
  that they will pursue to facilitate these goals.  Such opportunities
  may include, but are not limited to, teaching opportunities, specific types of training (in
  written and/or oral communication, team skills, particular research
  methods, etc.), internships, acting as a mentor, or going to
  conferences or workshops in particular subject areas.

\end{enumerate}

The written document must be sent to the dissertation committee at least two
weeks (10 business days) prior to their first formal dissertation committee
meeting.  The first formal dissertation committee meeting is open to the public,
and contains the following elements:

\begin{enumerate}
\item  A presentation covering the first three aspects of the written
  proposal, which should be approximately 45 minutes long and in the
  style of a research seminar.

\item  A public question and answer period moderated by the
  chairperson of the dissertation committee, which should be no longer than
  fifteen minutes long.  Members of the audience are allowed to ask
  questions; the dissertation committee should remain silent during this
  period.

\item  A private question and answer period comprised of only the
  student and their dissertation committee.  The main goals of this
  questioning are to determine the soundness and feasibility of the
  student's dissertation research, to ensure that the student
  understands the broader context within which their work is being
  done, and to determine the practicality of the student's
  professional development plan.  The committee is also expected to
  provide constructive feedback about all aspects of the student's
  written and oral presentations.

\end{enumerate}

\vspace{2mm}

\noindent
The possible outcomes of the comprehensive exam are:

\begin{enumerate}
\item  \textbf{Pass without qualification:} The student has
  successfully demonstrated all three of the goals described at the
  beginning of this section.

\item  \textbf{Pass with qualification:} The student has been found to
  be deficient in one of the goals described at the beginning of this
  section.

\item  \textbf{Fail:} The student has significant deficiencies in more
  than one of the goals described at the beginning of this section.

\end{enumerate}

If a student passes \textit{without} qualification, they have formally
passed their comprehensive examination requirement and no further
action is needed on their part.

If a student passes \textit{with} qualification, the dissertation committee
may specify that the student take action to remediate the observed
deficiency.  Such actions may include, but are not limited to, extra
practice with written and/or oral presentations; additional
coursework; additional reading of the literature, possibly with a
formal requirement to summarize their reading in written form for the
committee; or a revised research plan or professional development
plan.  This outcome does not require an additional committee meeting;
instead, the dissertation advisor is responsible for ensuring that the
student undertakes any remedial actions, and will report on this to
the dissertation committee at the next meeting.

Written feedback will be provided by the dissertation committee
regardless of the outcome of the meeting, reflecting the consensus of
the committee.  Optionally, individual members of the committee may
provide additional feedback.
If a student fails the comprehensive examination, their dissertation
committee will provide them, in writing, specific suggestions on
how they can improve their performance in the future.  These
suggestions will be part of the committee's report on the student's
progress.  The student must retake
the comprehensive examination within six months, and if they fail a
second time they will not be allowed to continue in the PhD program.


\vspace{3mm}
\subsubsection{Doctoral Dissertation}
\label{sec:phd_dissertation}

\vspace{3mm}
\noindent
\textbf{PhD dissertation advisor}

Graduate students are assigned an advisor when they are accepted into
the CMSE PhD program.  Students affiliated with a particular CMSE
faculty member, whether a teaching assistant, research assistant, or
having a fellowship, will be advised by that faculty member.  Students
who are not yet affiliated with a specific faculty member will be
temporarily advised by either the Director of Graduate Studies or
another member of the Graduate Curriculum Committee.  The most
critical role of the advisor is to assist the student in developing
their \href{https://login.msu.edu/?App=J3205}{PhD Program Plan}, which
lays out plans for their proposed coursework and professional
development efforts.  This should ideally be done within two semesters
of entering the program, and \textit{must} be done within four
semesters.  The student and their advisor should also collaborate  to create an
\href{http://caffe.grd.msu.edu/IDP}{Individual Development Plan},
which is a document that provides a planning process that helps
students to identify professional development needs, career
objectives, and to facilitate communication between mentees and their
mentors.  The Graduate School
\href{https://grad.msu.edu/prep}{provides resources} for developing an
\href{http://caffe.grd.msu.edu/IDP}{Individual Development Plan}, and
students are encouraged to consult these resources prior to meeting
with their mentor.

The graduate student should choose their official dissertation advisor
before the end of their second academic year in the program, and
ideally in the first academic year.  This requires mutual consent
between the professor and the student, and many factors go into this
important decision.  If the student has trouble in finding a willing
faculty member to serve as the their dissertation advisor, they should
consult the Graduate Director and/or the Department Chairperson to
help find a suitable match.  It is expected that students in the CMSE
PhD program will be funded by a fellowship or by a grant obtained by
their dissertation advisor by no later than the end of their second
year in the PhD program.

The dissertation advisor will be the chair of the student's PhD
guidance committee (see the next section) and, with the help of this
committee, will advise and mentor the student in their research and
professional development.  

Any faculty member with a non-0\% percentage appointment in the
Department of Computational Mathematics, Science and Engineering may
serve as an advisor for a CMSE doctoral dissertation.  Faculty members
without an appointment in CMSE may not serve as the sole dissertation
advisor for a student in the CMSE PhD program; they may, however,
co-advise a CMSE PhD student along with a CMSE faculty member.

If the dissertation advisor leaves Michigan State University before the
student completes their degree program, the student should consult
the Graduate Director and the Departmental Chairperson to identify a
suitable new advisor.  It is the joint responsibility of the
student and the Departmental Chairperson to make arrangements for the
completion of the degree, and it requires mutual consent between the
student and a new dissertation advisor.  The student's former
dissertation advisor may participate in their dissertation guidance
committee as an external, advisory (i.e., non-voting) member to help
ensure continuity in the student's research program.

If the student desires to change their dissertation advisor for any
reason, the change should be requested in writing as early as possible in their
graduate training program.  Any plans for changing to a different
advisor should be discussed with the Graduate Director, 
 the current dissertation advisor, and the student's
prospective new dissertation advisor (not necessarily together) prior
to the initiation of any change.  Before relations with the current
dissertation advisor are severed, the student should make sure that
another faculty member will serve in that capacity.  Research
Assistantships are normally associated with specific research programs
and are not automatically transferable from one faculty member to
another.

\vspace{3mm}
\noindent
\textbf{PhD dissertation guidance committee}

The student's dissertation committee should be formed by the end of the
summer following their first academic year in the PhD program, and
\textbf{must} be formed no later than the end of their fourth semester
in the program. At its early stages, this committee is primarily
intended to guide students in their choice of coursework; at later
stages of a student's PhD program they administer the comprehensive
exam (as the student's first formal dissertation committee meeting), 
monitor the progress of the dissertation, and give timely and
constructive feedback to the student on all aspects of their graduate
work and professional development.

A student's dissertation committee is composed of a minimum of four
members with the following requirements (beyond the requirements of
the \href{https://hr.msu.edu/documents/facacadhandbooks/facultyhandbook/composition.htm}{MSU graduate
college}):

\begin{itemize}
\item At least two members of the committee must be Michigan State
  University faculty members with a non-0\% appointment in CMSE, one
  of whom must be primarily in CMSE (i.e., CMSE must be their tenure
  home).

\item One committee member \textbf{must} be a MSU faculty whose appointment
  is entirely outside of CMSE (i.e., may not have any appointment in
  CMSE, including a 0\% appointment).

\end{itemize}

Members of external institutions or non-tenure stream faculty or
academic specialists may participate in a dissertation committee in an
official capacity as one of the four required faculty members, but
only if an exception is granted through approval first by the
Associate Dean for Graduate Education in the College of Engineering,
and then by the Dean of The Graduate School.  The majority of
committee members must, however, be tenure-stream MSU faculty.  The
Chairperson of the committee (who is the student's dissertation
advisor) must be an MSU faculty member with a non-0\% appointment in
CMSE.

The student must meet with their dissertation committee on an at least
annual basis to present on their progress.  Their dissertation
committee will report on the outcome of this meeting in writing to the
student and to the CMSE Graduate Director and the Graduate School,
with students having the ability to respond to this report (also in
writing).
The student must display satisfactory progress every year.  If the
student does not display satisfactory progress, the dissertation
committee should meet on a more rapid cadence to monitor and report on
their progress, including providing written documentation of
expectations for the student, quantifiable observations of the
student's progress, and consequences of failure to meet expectations
in the future.  If a student is deemed by the dissertation committee
to have made unsatisfactory progress for two dissertation committee
meetings in a row, with at least a three month separation between
meetings, the student will not be allowed to continue in the PhD
program.

\vspace{3mm}
\noindent
\textbf{Dissertation and dissertation defense}

The student's dissertation must be composed of novel research that
advances the state-of-the-art in algorithms or applications relating
to computational and/or data science.

A student may not submit their dissertation to their guidance
committee without approval of their advisor.  It is the dissertation
advisor's responsibility to ensure that the dissertation document
meets the university, college, and departmental standards for novel
research, that it further represents a contribution that is
significant enough to merit the receipt of a PhD, and that the
document is well-written and  conforms to departmental norms in
terms of structure and content.  More broadly, it is the advisor's
responsibility to ensure that a student does not attempt to defend
their dissertation until the student is ready and  their dissertation
is likely to be approved by their guidance committee.

A student's PhD dissertation must conform to MSU formatting standards
(a \href{http://ctan.org/pkg/msu-thesis}{MSU \LaTeX\ thesis class
  template} is available), and their submission-ready dissertation should be submitted to the entire
dissertation committee electronically no less than two weeks (10 business days) prior to
their dissertation defense.  The same version of the dissertation will also be available for
viewing to the entire department electronically.

The dissertation defense is an oral defense that has the following
components:

\begin{enumerate}

\item An oral presentation, open to the public, where the student
  presents background material and the key findings of their
  dissertation in the style of a research seminar.  This presentation
  should be approximately 45 minutes long, and does not need to
  comprehensively cover the student's dissertation results (i.e., a
  student can choose to focus on some subset of topics from their
  dissertation if they choose).  The content of this presentation
  should be chosen in consultation with the student's dissertation advisor.

\item A public question and answer period following their oral
  presentation, wherein the student answers questions posed by
  non-committee members.  The chair of the dissertation committee may
  moderate these questions and is encouraged to limit them to no more
  than an additional half hour.

\item A private session where the student answers questions from the
  committee regarding their dissertation research.  At the committee's
  discretion, they may also question the student about the
  fundamentals of the algorithms or application area relating to the
  student's dissertation research.

\end{enumerate}

The possible outcomes of the presentation of the disssertation and oral defense can be:

\begin{enumerate}

\item \textbf{Pass with no revisions.}  The dissertation and oral
  defense both meet or exceed the standards for quality and original
  research expected of students in the CMSE program, and the dissertation
  committee needs no changes to be made to the dissertation.  The
  student may immediately deposit their dissertation with the Graduate
  School.

\item  \textbf{Pass with minor revisions.}  The dissertation and oral
  defense both meet or exceed the standards  for quality and original
  research expected of students in the CMSE program, but the dissertation
  committee deems minor changes to be made to the dissertation
  document (e.g., typos corrected, the addition of small amounts of
  clarifying text, etc.).  A written list of the requested changes
  will be created by the dissertation committee and presented to the
  student, who then must make changes to their dissertation as
  necessary.  The student's dissertation advisor(s) will make the
  final decision about acceptance of the written dissertation.
  Following acceptance by the advisor, the student may deposit their
  dissertation with the Graduate School.

\item  \textbf{Pass with major revisions.}  The dissertation and oral
  defense both meet the standards for quality and original research
  expected of students in the CMSE program, but the dissertation committee
  deems major revisions must be made to the dissertation.  In this
  context, ``major changes'' mean the addition or removal of significant
  amounts of text or pursuing additional research/(re-)analysis and
  adding that to the dissertation document.  A written list of the
  requested changes will be created by the dissertation committee and
  presented to the student, who then must make changes to their
  dissertation as necessary.  After the student has provided the
  entire dissertation committee with a revised version of their dissertation
  (within a reasonable period of time), the dissertation must be accepted by
  a majority of members of the dissertation committee, one of whom
  \textbf{must} be the student's dissertation advisor.  Following acceptance
  by the majority of the dissertation committee, the student may deposit
  their dissertation with the Graduate School.

\item  \textbf{Failure.} The dissertation and/or oral defense have one
  or more irredeemable flaws and the majority of the dissertation
  committee (which may, but does not necessarily have to include, the student's dissertation advisor)
  agrees that an acceptable level of quality cannot be obtained in a
  reasonable length of time, then the dissertation will be rejected
  and the student will be removed from the PhD program.


\end{enumerate}

\noindent
After the successful completion of the oral exam and, if necessary,
revisions to the dissertation, students are responsible for submitting
their dissertation to the Graduate School along with a Dissertation
Approval Form.  Complete
instructions \href{https://grad.msu.edu/etd}{can be found here}.

% \vspace{3mm}
% \subsubsection{Responsible Conduct of Research Training}
% \label{sec:rcr}

% All PhD students must complete Responsible Conduct of Research
% Training to fulfill the requirements specified by the \href{http://grad.msu.edu/rcr/}{Graduate
% School} and administered by the College of Engineering, which addresses requirements regarding
% graduate training from the National Science Foundation and National
% Institute of Health, among other federal agencies.  This training can
% include online training offered by the \href{https://www.egr.msu.edu/secureresearchcourses/}{College of
% Engineering}  or one of
% any number of in-person training sessions.

\vspace{3mm}
\subsubsection{Annual Progress Report}

All students in the CMSE PhD program are expected to submit a
\href{https://www.egr.msu.edu/academics/graduate/graduate-student-annual-reporting-requirements}{Graduate
  Student Annual Report}, which is due by January 31st of each year.
To quote the linked website:  ``As part of this report, students will
report their progress during the previous year, review their academic
and professional goals, and communicate with their adviser(s) about
their plans and progress toward degree completion. PhD students who do
not complete the annual reporting process will have a hold placed on
their accounts.''  Students will receive written feedback on this progress
report from their academic advisor and/or the Graduate Director within
a month of submission of the report.

\vspace{3mm}
\subsubsection{Additional notes}

For an explanation of how to obtain a dual PhD in CMSE and a secondary
subject, please see Section~\ref{sec:dual_phd}.




