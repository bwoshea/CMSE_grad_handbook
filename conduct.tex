\section{Student conduct and conflict resolution}

\subsection{Student Conduct}

Graduate students are an integral and highly valued part of the
department's research and teaching programs.  Professional behavior is
expected from all CMSE graduate students at all times.  In this
context, professional behavior has several key aspects.  You are
expected:

\begin{itemize}

\item To work responsibly toward completion of your chosen degree in a
timely fashion.

\item To contribute to your scholarly discipline by learning the
theoretical and practical aspects of your chosen field of study, by
constructing new knowledge, and by applying that knowledge to new
problems.

\item To exercise the highest integrity in all aspects of your work at
Michigan State University, particularly with regards to research and
teaching.

\item To treat fellow students, staff, and faculty with the courtesy
and respect with which you would like to be treated in order to create
an environment of collegiality and collaboration.  In particular, this
means avoiding any inappropriate behavior that may be interpreted as
harassment.

\item To devote the same seriousness to instructional duties (both
undergraduate and graduate) that you would expect from your own
instructors.

\end{itemize}

\noindent
In addition to these broad expectations regarding your conduct while
at Michigan State University, please consult the
\href{http://splife.studentlife.msu.edu/}{Spartan Life Student
Handbook}, which has specific policies, ordinances, and regulations
that define additional University expectations.  Further important
policy documents that pertain to graduate student conduct are the
\href{http://splife.studentlife.msu.edu/academic-freedom-for-students-at-michigan-state-university}{MSU
Student Rights and Responsibilities} document, as well as the MSU
Graduate School's \href{https://grad.msu.edu/gsrr}{Graduate Student
Rights and Responsibilities} document.  Please note that failure to
conform to the expected level of professionally accepted behavior may
result in dismissal from the graduate program.

\subsection{Conflict Resolution}

Occasionally problems involving students, teaching assistants,
research assistants, staff, and faculty arise.  Many of these problems
are likely to be resolved by informal discussions with the Graduate
Director or with the Department Chair.  You are encouraged to contact
these individuals with any issues that may arise, starting with the
Graduate Director.  If the Graduate Director is the person you have a
problem with, or you are otherwise uncomfortable discussing the issue
with them, you should go directly to the Department Chair.  If
students are working with faculty in one of these administrative
positions, they should contact the Associate Department Chair.  If
they are working with faculty who occupy all of these administrative
positions, they should contact the office of the Associate Dean of Graduate Studies
in the College of Engineering, as described in the grievance policy below.

In rare cases, some conflicts may need a more formal
mechanism for resolution.  Graduate students in the Department of
Computational Mathematics, Science and Engineering are officially in
the College of Engineering.  The grievance procedure for College of
Engineering graduate students is described by
\href{https://www.egr.msu.edu/sites/default/files/content/GRAD/2015\%20Grievance\%20Hearing\%20Procedure.pdf}{this
document}, which is based on the template provided by the
\href{https://ombud.msu.edu/}{University
Ombudsperson}.  This template is in accordance with the agreement
between the  Graduate Employee Union and Michigan State University, as
described in detail in the \href{https://www.hr.msu.edu/documents/contracts/GEU2015-2019.pdf}{MSU/Graduate
  Employee Union Contract}.


% Some key points regarding Hearing Board procedures for academic
% grievences, taken from the document linked above, are as follows:

% \begin{itemize}

% \item The Hearing Board serves as the initial Hearing Board for
% academic grievance hearings involving graduate students who allege
% violations of academic rights or seek to contest an allegation of
% academic misconduct (academic dishonesty, violations of professional
% standards or falsifying admission and academic records).

% \item The Hearing Board is composed of equal numbers of faculty and
% graduate students, and will be chaired by the Departental Chairperson
% (who will only vote in the occasion of a tie).  If the Department
% Chairperson is involved in the case, the Graduate Director or another
% senior member of the faculty will act as the chair of the Hearing
% Board.

% \item After consulting with the appropriate unit administrator
% (Graduate Director or Department Chairperson), graduate students who
% remain dissatisfied with their attempt to resolve an academic
% grievance may request a hearing. When appropriate, the Department
% Chair may waive jurisdiction and refer the request for an initial
% hearing to the College Hearing Board.

% \item At any time in the grievance process, either party may consult
% with the University Ombudsperson.

% \item Hearings must be requested in writing and will be held promptly,
% using a set of procedures designed to provide full opportunity for
% explanations, questions, and rebuttals.

% \item Either party to a grievance may appeal the decision of the
% Department hearing board to the Engineering College hearing board. All
% appeals must be in writing.

% \end{itemize}


