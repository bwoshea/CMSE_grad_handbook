\section{M.S. in CMSE}
\label{sec:ms}

\subsection{Program overview}


The overall goal of the M.S. in CMSE is to give students broad and
deep knowledge of the fundamental techniques used in computational
modeling and data science, as well as significant exposure to at least
one application domain.

Students that have completed this M.S. program will be able to:

\begin{itemize}
\item  Analyze problems in terms of the algorithms and pre-existing
  computational tools required to solve a range of problems in
  computational and data science, and write programs to efficiently
  solve the problem using cutting-edge computational hardware.  

\item  Construct and implement models and simulations of physical,
  biological, and social situations, and use these models/simulations
  to understand experimental or observational data.  

\item  Apply discipline-focused or methodology-focused topics in
  computational and data science to solve problems in the student’s
  application domain of choice.

\item  \red{\textbf{OTHER THINGS HERE?}}

\end{itemize}

\subsection{Program components}

Students pursuing a M.S. in CMSE must complete a minimum of 30 credits
of graduate coursework, at least 18 credits of which must be CMSE
courses. There are two possible `tracks' in the program: Plan A, which
includes a thesis, or Plan B, which does not have a thesis. The
student’s program of study must be approved by the student’s guidance
committee. The student must meet the requirements specified below:

\vspace{3mm}
\noindent
\textbf{Common Requirements for Plan A and Plan B:}

\begin{enumerate}

\item Complete a minimum of three of the following four core CMSE
  courses \textbf{with an average grade of at least 3.3 on the
    corresponding subject exams and no grade lower than 3.0} (at least
  9 credits):

\begin{itemize}
    \item  CMSE 820, Mathematical Foundations of Data Science (3 credits)  
    \item  CMSE 821, Numerical Methods for Differential Equations (3 credits)  
    \item  CMSE/CSE 822, Parallel Programming (3 credits)  
    \item  CMSE 823, Numerical Linear Algebra, I (3 credits)  
\end{itemize}

\item Complete a minimum of 12 credits in complementary coursework
  chosen in consultation with, and approved by, the student’s academic
  advisor. 
 
\item All M.S. students must complete Responsible Conduct of Research
  Training.
\end{enumerate}

\vspace{3mm}
\noindent
\textbf{Additional Requirements for Plan A:} In addition to
requirements 1-4 above, the student must complete a thesis based on
original research on a problem in computational and/or data
science. The student will enroll in a minimum of 4 and a maximum of 8
credits of CMSE 899 (Master’s Thesis Research). This thesis research
will culminate in a written thesis to be submitted to, and accepted
by, a guidance committee. Part of the acceptance process may include
an oral examination of the student’s work.

\vspace{3mm}
\noindent
\textbf{Additional Requirements for Plan B:} Requirements 1-4 will
apply. In lieu of pursuing original research in computational and/or
data science, the student will enroll in additional coursework. This
coursework may include up to 3 credits of CMSE 891 (Independent Study)
if approved by the student’s academic advisor. There is no requirement
for either a written thesis or oral presentation.

\subsection{Admission requirements}

At present, admission directly to the Master of Science degree is only
available in exceptional circumstances.  The admissions criteria are
identical to those of the CMSE PhD program, as described in
Section~\ref{sec:phd_admission}.

\subsection{Degree requirements}

\subsection{Selection of thesis advisor}

\subsection{Formation of the guidance committee}

\subsection{Thesis defense and final oral examination}
