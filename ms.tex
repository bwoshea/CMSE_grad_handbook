\section{Master of Science in CMSE}
\label{sec:ms}

\subsection{Program overview}

The overall goal of the Master of Science program in CMSE is to give students broad and
deep knowledge of the fundamental techniques used in computational
modeling and data science, as well as significant exposure to at least
one application domain.

\vspace{2mm}
\noindent
Students that have completed this M.S. program will be able to:

\begin{itemize}
\item  Analyze problems in terms of the algorithms and pre-existing
  computational tools required to solve a range of problems in
  computational and data science, and write programs to efficiently
  solve the problem using cutting-edge computational hardware.  

\item  Construct and implement models and simulations of physical,
  biological, and social situations, and use these models/simulations
  to understand experimental or observational data.  

\item  Apply discipline-focused or methodology-focused topics in
  computational and data science to solve problems in the student’s
  application domain of choice.

\item  \red{\textbf{OTHER THINGS HERE?}}

\end{itemize}

\subsection{Program requirements}

Students pursuing a M.S. in CMSE must complete a minimum of 30 credits
of graduate coursework, at least 18 credits of which must be CMSE
courses. There are two possible `tracks' in the program: Plan A, which
includes a thesis, or Plan B, which does not have a thesis. The
student’s program of study must be approved by the student’s guidance
committee. The student must meet the requirements specified below:

\vspace{3mm}
\noindent
\textbf{Common Requirements for Plan A and Plan B:}

\begin{enumerate}

\item Complete a minimum of three of the following four core CMSE
  courses \textbf{with an average grade of at least 3.3 on the
    corresponding subject exams and no grade lower than 3.0} (at least
  9 credits):

\begin{itemize}
    \item  CMSE 820, Mathematical Foundations of Data Science (3 credits)  
    \item  CMSE 821, Numerical Methods for Differential Equations (3 credits)  
    \item  CMSE/CSE 822, Parallel Programming (3 credits)  
    \item  CMSE 823, Numerical Linear Algebra, I (3 credits)  
\end{itemize}

\item Complete a minimum of 12 credits in complementary coursework
  chosen in consultation with, and approved by, the student’s academic
  advisor.  (Note: the academic advisor for M.S. students is typically the CMSE Graduate Director.)
 
\item All M.S. students must complete Responsible Conduct of Research
  Training.
\end{enumerate}

\vspace{3mm}
\noindent
\textbf{Additional Requirements for Plan A:} In addition to
requirements 1-4 above, the student must complete a thesis based on
original research on a problem in computational and/or data
science. The student will enroll in a minimum of 4 and a maximum of 8
credits of CMSE 899 (Master’s Thesis Research). This thesis research
will culminate in a written thesis to be submitted to, and accepted
by, a guidance committee. Part of the acceptance process may include
an oral examination of the student’s work.

\vspace{3mm}
\noindent
\textbf{Additional Requirements for Plan B:} Requirements 1-4 will
apply. In lieu of pursuing original research in computational and/or
data science, the student will enroll in additional coursework. This
coursework may include up to 3 credits of CMSE 891 (Independent Study)
if approved by the student’s academic advisor. There is no requirement
for either a written thesis or oral presentation.

\subsection{Admission requirements}

At present, admission directly to the Master of Science degree is only
available in exceptional circumstances.  The admissions criteria are
identical to those of the CMSE doctoral program, as described in
Section~\ref{sec:phd_admission}.

\subsection{Selection of thesis advisor}

Any faculty member with a non-zero percentage appointment in the
Department of Computational Mathematics, Science and Engineering may
serve as an advisor for a Master's thesis for a student pursuing Plan
A.  With the permission of the CMSE Graduate Director and CMSE
Graduate Studies Committee, a faculty member without an appointment in
CMSE may serve as a Master's thesis advisor.

\subsection{Formation of the guidance committee}
\label{sec:ms_guidance_comm}

The purpose of the M.S. guidance committee is primarily to provide
advice to students about coursework selection and professional
development.  In the case of students pursuing a Plan A Master's
degree, the guidance committee is also responsible for advising the
student in their choice of research topic and accepting the written
thesis and, optionally, providing an oral defense of the thesis.

All students pursuing a Master's degree in CMSE will be assigned a
two-person guidance committee upon entrance to the degree program.
This committee will be comprised of the CMSE Graduate Director and a
member of the CMSE Graduate Studies Committee, with some effort made
to ensure that at least one of these faculty members has research
interests that are in the general area of those of the Master's
student to ensure that reasonable advice is given.

Students who choose to pursue a Plan A (i.e., thesis-based) Master's
degree must form a more targeted guidance committee prior to starting
to take CMSE 899 (Master's thesis research).  This committee may be
comprised of any three MSU faculty, as long as at least two of the
faculty members have appointments in CMSE and one has their majority
appointment (i.e., tenure home) in CMSE.

Students who choose to pursue a Plan B (non-thesis-based) Master's
degree have no additional requirements regarding their guidance
committee.

\subsection{Thesis defense and final oral examination}

Students pursuing a Plan A thesis submit a written thesis to their
guidance committee for approval four weeks before the final date for
thesis deposition during their last semester in the Master's degree
program.  (Note: this date can be found in the
\href{https://reg.msu.edu/ROInfo/Calendar/Academic.aspx}{MSU Academic
Calendar}, but is typically the Monday after the end of final exam
week.)  Submission should be electronic, and preferably via PDF.
Students who wish to use \LaTeX should use the
\href{http://ctan.org/pkg/msu-thesis}{MSU \LaTeX thesis class}, which
will ensure that their thesis conforms to the
\href{https://grad.msu.edu/etd/formatting-guide}{Graduate School
Thesis Formatting Guide}.

The guidance committee will give feedback on the thesis within two
weeks, and the student is expected to make any necessary revisions and
get approval from their guidance committee in a timely manner.  After
their guidance committee approves of the thesis, it must be submitted,
along with an Approval Form, to the Graduate College.  Complete
instructions \href{https://grad.msu.edu/etd}{can be found here}.

No oral examination is required for the Plan A Master's degree.  If a
student requests one, the same procedure as is used for the PhD oral
defense shall be used.  See Section~\ref{sec:phd_dissertation} for
additional details.