\section{Graduate Certificate in Computational Modeling}
\label{sec:cert_model}

\subsection{Certificate description}

The Graduate Certificate in Computational Modeling is intended for
students with little or no prior programming or computational modeling
experience. The purpose of this certificate is to complement graduate
students' degree programs with a set of courses that teach students
critical skills in computer programming, data manipulation and
visualization, and computational modeling.  

Students that have completed this certificate will be able to: 

\begin{itemize} 
\item  Demonstrate a basic understanding of functional computer
  programming as applied to a range of problems in computational and
  data science.  

\item  Analyze problems in terms of the algorithms and pre-existing
  computational tools required to solve a range of problems in
  computational and data science, and write a program to efficiently
  solve the problem.  

\item  Construct and implement models and simulations of physical,
  biological, and social situations, and use these models/simulations
  to understand experimental or observational data.  

\item  Apply some subset of discipline-focused or methodology-focused
topics in computational and data science to solve problems in the
student’s primary discipline.

\end{itemize}

\subsection{Certificate requirements}

he Graduate Certificate in Computational Modeling consists of at least
three courses comprising a minimum of 9 credit hours, taken from the
two categories listed below.  The targets of the certificate program
are graduate students in any discipline with interest in applying
computational and data science approaches to their research problems,
or who generally desire a broad education in computational modeling
and computational methodology.  To facilitate this goal, in addition
to there being no restriction on graduate student discipline, students
can apply for the certificate at any time prior to receiving their
degree (either Master’s or PhD), and can apply for the certificate
after taking all the necessary courses. 

\begin{enumerate}
\item Any two of the CMSE core graduate courses (6 credits):  

\begin{itemize}
    \item  CMSE-801, Introduction to Computational Modeling (3 credits)
    \item  CMSE-802, Methods in Computational Modeling (3 credits)  
    \item  CMSE-820, Mathematical Foundations of Data Science (3 credits)  
    \item  CMSE-821, Numerical methods for differential equations
    \item  CMSE/CSE-822, Parallel programming (3 credits) 
    \item  CMSE-823, Numerical Linear Algebra, I (3 credits)
\end{itemize}

\item One or more additional courses, which may include further CMSE
  courses at the 400 level or above (including from the list of core
  CMSE graduate courses in List 1), courses from the list of non-CMSE
  courses  in Section~\ref{sec:courses}, or other computational
  science or data science-focused courses at the 400 level or above as
  approved by the CMSE graduate advisor (3 or more credits).

\end{enumerate}

\subsection{Admission and graduation requirements}

Graduate students enrolled at Michigan State University in any
discipline or college may pursue this graduate certificate.
Furthermore, students can apply for the certificate at any time prior
to receiving their primary degree (either Master’s or PhD), and can
apply for the certificate after taking all the necessary courses.

Graduate students pursuing either the Master of Science in CMSE, the
Doctor of Philosophy in CMSE, or a dual PhD in CMSE and a second
subject \textbf{may not also receive} a graduate certificate in
Computational Modeling.

In order to obtain this graduate certificate the student must have at
least a 3.0 average in the courses that are applied to the
certificate.

