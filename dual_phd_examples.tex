\section{Examples of dual PhD programs of study}
\label{sec:dual_phd_examples}

This section provides examples of dual PhD programs of study
incorporating CMSE and a second subject for several of the programs
that are common partners in this endeavor.  This information is
provided as guidance for students developing a dual PhD program, and
is non-binding (i.e., it does not represent a set of
\textit{requirements}, simply a set of guidelines).  Final approval of
any dual PhD program rests with the student's dissertation committee
as well as the graduate directors of CMSE and the second PhD program.

The requirements of the CMSE PhD program is described in detail in
Section~\ref{sec:phd}, and the Department of CMSE's guidelines on dual
PhDs with CMSE and a second subject are described in
Section~\ref{sec:dual_phd}.

Note that, regardless of the other department's requirements, the
Department of Computational Mathematics, Science and Engineering
requires that a dual PhD program must form their dissertation
committee (or a guidance committee whose composition approximates
their dissertation committee) prior to the end of their second year in
a MSU PhD program, discuss their planned program of study with this
committee, and ensure that all of the proper paperwork has been
fulfilled by the end of their second year.

%%%%%%%%%%%%%%%%%%%%%%%%%%%%%%%%%%%%%%%%%%%%%%%%%%%%%%%%%%%%%%%%
\subsection{Mathematics}

The mathematics graduate handbook can be
\href{https://math.msu.edu/graduate/files/handbook/Graduate_Student_Handbook.pdf}{found
  here}, and explicitly contains the Department of Mathematics
requirements for a dual PhD in mathematics and a second subject
(section IV.C).

In order to receive a PhD in Mathematics, a student must: (1) satisfy
the qualifying examination requirements by passing written exam
corresponding to 3 of 5 areas of specialization within one year of
entrance in the PhD program (and taking the two-semester course
sequences for these areas, totaling 18 credits, if
the student does not ``pass out'' of the exam in the August exam
period before their initial year);
(2) Pass the comprehensive examination in the area of the
student's research interests, which includes both a written and oral
exam; (3) take thirty credits of 800-900 level mathematics courses,
excluding dissertation credits (MTH 999) and core courses in areas in
which the qualifying examination requirements are fulfilled; (4) take
24 dissertation credits (MTH 999); and, (5) write and defend a
doctoral dissertation acceptable to the student's dissertation
committee.

\subsubsection{CMSE as the primary program}

To fulfill the qualifying exam requirements in mathematics, candidates
whose secondary program is mathematics must pass two mathematics
qualifying exams (rather than three) within the first three years of
the student’s enrollment at Michigan State University.  They must pass
three of the CMSE subject exams within their first two years to
fulfill the CMSE qualifying exam requirement.  Note that the
Department of Mathematics requires that ``The topics covered in the
qualifying exams taken by a dual degree candidate must differ
substantially for each other. This includes all qualifying exams in
both subject areas. For example, a dual degree candidate in
mathematics and CMSE may not take for credit both the mathematics
qualifying exam in numerical analysis and the CMSE qualifying exam in
numerical analysis.''  Students must take 15 credits of 800-900 level
mathematics courses, excluding dissertation credits (Math 999) and
qualifying exam course sequences, and must also take additional
computationally-focused courses to satisfy the CMSE course
requirement.  The Department of Mathematics requires that all
candidates with dual PhDs in mathematics must fulfill their
comprehensive examination requirement as specified for PhD candidates
in mathematics, any time after the qualifying exam requirements have
been met and prior to the end of the student's fourth year at MSU.
Furthermore, the comprehensive exam can include topics and questions
from the dual program to allow the comprehensive exam to satisfy the
requirements of the dual program, although at least half the topics
and questions should be in mathematics.  A further restriction is that
the topics covered by the comprehensive exam must be approved by the
Mathematics graduate director and Mathematics graduate studies
committee.  To satisfy the CMSE comprehensive exam requirements,
students must write a comprehensive exam paper containing the same
material as a student fully within the CMSE PhD program, as described
in Section~\ref{sec:comp_exam}.  Regarding committee members, the
Mathematics requirement is as follows: ``For secondary candidates the
guidance committee and dissertation committee should consist of four
or more tenure stream faculty members at least 40\% of whom have a
50\% or more appointment in Mathematics.''  The Department of CMSE
requires that at least one faculty member have their tenure home in
CMSE (i.e., more than 50\% of their appointment in CMSE).
Dissertation credits should be in CMSE (i.e., CMSE 999).

\subsubsection{Mathematics as the primary program}

According to the Mathematics guidelines, students pursuing a dual PhD
with mathematics as the primary program must fulfill the qualifying
course and exams requirements exactly as specified for PhD candidates
in mathematics and in the same time frame.  In addition, these
students must pass two of the CMSE subject exams within their first
two years.  Note that the Department of Mathematics requires that
``The topics covered in the qualifying exams taken by a dual degree
candidate must differ substantially for each other. This includes all
qualifying exams in both subject areas. For example, a dual degree
candidate in mathematics and CMSE may not take for credit both the
mathematics qualifying exam in numerical analysis and the CMSE
qualifying exam in numerical analysis.''  Students must take 21
credits of 800-900 level mathematics courses, excluding dissertation
credits (Math 999) and qualifying exam course sequences.  Students
must take a minimum of 12 credits of coursework in
computationally-focused subjects to satisfy their CMSE course
requirements, which may be satisfied in part by computationally-focused
mathematics courses (but may not be satisfied by the Mathematics
qualifying exam sequence courses).  The Department of Mathematics requires that all
candidates with dual PhDs in mathematics must fulfill their
comprehensive examination requirement as specified for PhD candidates
in mathematics, any time after the qualifying exam requirements have
been met and prior to the end of the student's fourth year at MSU.
Furthermore, the comprehensive exam can include topics and questions
from the dual program to allow the comprehensive exam to satisfy the
requirements of the dual program, although at least half the topics
and questions should be in mathematics.  A further restriction is that
the topics covered by the comprehensive exam must be approved by the
Mathematics graduate director and Mathematics graduate studies
committee.  To satisfy the CMSE comprehensive exam requirements,
students must write a comprehensive exam paper containing the same
material as a student fully within the CMSE PhD program, as described
in Section~\ref{sec:comp_exam}.  Regarding committee members, the
Mathematics requirement is as follows: ``For primary candidates the
guidance committee and dissertation committee should consist of four
or more tenure stream faculty members at least half of whom have a
50\% or more appointment in Mathematics.''  The Department of CMSE
requires that at least one faculty member have their tenure home in
CMSE (i.e., more than 50\% of their appointment in CMSE).
Dissertation credits should be in Mathematics (i.e., MTH 999).


%%%%%%%%%%%%%%%%%%%%%%%%%%%%%%%%%%%%%%%%%%%%%%%%%%%%%%%%%%%%%%%%
\subsection{Statistics and Probability}

Information on the Statistics PhD program can be
\href{https://www.stt.msu.edu/Graduate_Program/handbook/PHDREQ.pdf}{found
  here}.

In order to receive a PhD in Statistics and Probability, a student
must (1) take the 5 core STT courses (STT 872, 881-2, 867-8; 15
credits total), (2) take at least 5 courses in advanced probability or
statistics, with at least one in each category (15 credits), (3) take
3 elective courses (9 credits), which may be from outside the
department, (4) pass the Probability and Statistics preliminary
examinations within three years of being admitted to the program, and
(5) complete an original dissertation and enroll in at least 24
credits of STT 999.  The PhD program in Statistics and Probability has
no comprehensive exam requirement.

\subsubsection{CMSE as the primary program}

Students pursuing a dual PhD with CMSE as the primary program must
take STT 872 and 881, as well as any three of the four core CMSE
courses (CMSE 820-823).  They must pass one of the Statistics
preliminary exams and at least three of the CMSE subject exams.  They
must further take 5 elective courses, with at least 2 from STT (with
the STT courses chosen from the remaining core STT and/or advanced
probability and statistics courses, and CMSE requirements as specified
elsewhere in this handbook).  Students must take the CMSE
comprehensive examination.  No particular requirements are specified
regarding guidance committee composition; however, at least one member
of the committee must have their tenure home in each of the two
departments.

\subsubsection{Statistics as the primary program}

Students pursuing a dual PhD with statistics as the primary program
must take STT 872, 881, and one course selected from STT 882, 867,
868, as well as any two of the four core CMSE courses (CMSE 820-823).
They must pass at least one of the Statistics preliminary exams and at
least two of the CMSE subject exams.  They must further take 5
elective courses, with at least 3 from STT (with the STT courses
chosen from the remaining core STT and/or advanced probability and
statistics courses, and CMSE requirements as specified elsewhere in
this handbook).  Students must take a minimum of 12 credits of
coursework in computationally-focused subjects to satisfy their CMSE
course requirements, which may be satisfied in part by computationally-focused
statistics courses (e.g., STT 874).  Students must take the CMSE comprehensive
examination.  No particular requirements are specified regarding
guidance committee composition; however, at least one member of the
committee must have their tenure home in each of the two departments.


%%%%%%%%%%%%%%%%%%%%%%%%%%%%%%%%%%%%%%%%%%%%%%%%%%%%%%%%%%%%%%%%
\subsection{Physics}

The Physics and Astrophysics graduate handbook can be \href{http://www.pa.msu.edu/grad/GradHandbook.pdf}{found
  here}.  
To receive a PhD in physics, students must: (1) pass the physics
qualifying examination, (2) complete a set of 7 core physics courses
(PHY 810, 820, 831, 851-2, 841-2), (3) fulfill the comprehensive exam
requirement by receiving acceptable grades in the subject exams
associated with PHY 820, 831, 841, and 851-2, (4) complete an original
dissertation and enroll in at least 24 credits of PHY 999, and (5)
serve as a Teaching Assistant (TA) for at least one semester.

\subsubsection{CMSE as the primary program}

Students pursuing a dual PhD with CMSE as the primary program must
take 3 of the 4 core CMSE courses and pass the corresponding subject
exams and take 2 of the 4 core PHY courses (or course sequences) and
pass the corresponding subject exams.  The remainder of the student's
coursework requirement can be fulfilled by any reasonable combination
of PHY and CMSE coursework, with students strongly advised to take
additional physics courses.  Students must take the CMSE comprehensive
examination.  The dissertation committee consists of five members with
at least two having their tenure homes in the Physics and Astronomy
department (with no requirement as to the members' sub-discipline) and
at least one member having their tenure home in CMSE.  Students with
CMSE as their primary program do not have a requirement that they
serve as a teaching assistant.  These students should take CMSE
dissertation credits (CMSE 999).

\subsubsection{Physics as the primary program}

Students pursuing a dual PhD with Physics as the primary program must
take 3 of the 4 core PHY courses (or course sequences) and pass the
corresponding subject exams and take 2 of the 4 core CMSE courses and
pass the corresponding subject exams.  The remainder of the student's
coursework requirement can be fulfilled by any reasonable combination
of PHY and CMSE coursework, with students strongly advised to take
significant additional physics courses.  Students must take a minimum
of 12 credits of coursework in computationally-focused subjects to
satisfy their CMSE course requirements, which may be satisfied in part
by
computationally-focused physics courses.  Students must take the CMSE
comprehensive examination.  The dissertation committee consists of
five members with at least three having their tenure homes in the
Physics and Astronomy department (with one being from outside the
student's interest area) and at least one member having their tenure
home in CMSE.  These students should take Physics dissertation credits
(PHY 999).

%%%%%%%%%%%%%%%%%%%%%%%%%%%%%%%%%%%%%%%%%%%%%%%%%%%%%%%%%%%%%%%%
\subsection{Astrophysics}

The Physics and Astrophysics graduate handbook can be 
\href{http://www.pa.msu.edu/grad/GradHandbook.pdf}{found
  here}.  

To receive a PhD in astrophysics, students must take a total of eight
courses comprising the four core astronomy courses (AST 810, AST 825,
AST 835, and AST 840), two of the physics subject exam courses
(PHY820, PHY 831, PHY 841, and PHY 851), and two additional courses (6
credits) selected from the core physics, astrophysics, or
computational courses.  They must complete a ``second year project,''
which culminates in a presentation to a committee of three faculty
that may be expanded into a dissertation committee.  Furthermore, they
must 
complete an original
dissertation and enroll in at least 24 credits of PHYAST 999, and
serve as a Teaching Assistant (TA) for at least one semester.

\subsubsection{CMSE as the primary program}

Students pursuing a dual PhD with CMSE as the primary program must
take 3 of the 4 core CMSE courses and pass the corresponding subject
exams and take 2 of the 4 core AST courses (or course sequences) and
pass the corresponding subject exams.  The remainder of the student's
coursework requirement can be fulfilled by any reasonable combination
of AST, PHY, and CMSE coursework, with students strongly advised to
take at least one of the physics courses.  Students must take the CMSE
comprehensive examination, the requirements of which should be
combined with the astrophysics second year project (or taken
separately, at the discretion of the dissertation committee).  The
dissertation committee consists of five members with at least two
having their tenure homes in the Physics and Astronomy department
(with no requirement as to the members' sub-discipline) and at least
one member having their tenure home in CMSE.  Students with CMSE as
their primary program do not have a requirement that they serve as a
teaching assistant.  These students should take CMSE dissertation
credits (CMSE 999).

\subsubsection{Astrophysics as the primary program}

Students pursuing a dual PhD with Astrophysics as the primary program
must take 3 of the 4 core AST courses and take 2 of the 4 core CMSE
courses and pass the corresponding subject exams.  The remainder of
the student's coursework requirement can be fulfilled by any
reasonable combination of AST, PHY, and CMSE coursework, with students
strongly advised to take at least one of the physics courses.  Students must take a minimum
of 12 credits of coursework in computationally-focused subjects to
satisfy their CMSE course requirements, which may be satisfied in part
by
computationally-focused astrophysics courses.
Students must take the CMSE comprehensive examination, the
requirements of which should be combined with the astrophysics second
year project (or taken separately, at the discretion of the
dissertation committee).  The dissertation committee consists of five
members with at least three having their tenure homes in the Physics
and Astronomy department (with one being from outside the student's
interest area) and at least one member having their tenure home in
CMSE.  These students should take Astrophysics dissertation credits
(AST 999).

