\section{Dual PhD in CMSE and a second subject}
\label{sec:dual_phd}

The Department of Computational Mathematics, Science and Engineering
strongly supports interdisciplinary PhD programs centering on the
student's pursuit of a project that combines a specific application or
algorithmic domain and the goals of the CMSE PhD program.  In order to
qualify for such a program, the student's dissertation must include
significant research contributions in both disciplines.

MSU allows ``dual PhD'' programs for individual students to span
graduate programs, as long as the graduate programs involved agree to
do so - see the
\href{https://reg.msu.edu/AcademicPrograms/Text.asp?Section=111#s407}{MSU
  guidelines on dual major doctoral degrees} for more information.  It
is typical that a student enters into a dual PhD program after
starting graduate school at MSU in their primary graduate program, and then
arranges the secondary affiliation upon choice of a research project
and advisor; however, a student could in principle be admitted as a
dual PhD student with concurrence of the two graduate programs.

The Department of Computational Mathematics, Science and Engineering
has developed a set of guidelines for these dual major PhDs, which
should apply to all students wishing to pursue a PhD jointly between
CMSE and another program.  These guidelines are as follows:

\begin{enumerate}

\item  A request for the dual major degree must be submitted for
  approval to the Graduate Directors of both departments and the Dean
  of the Graduate School within one semester following its development
  and within the first two years of the student’s enrollment at
  Michigan State University.  A copy of the guidance committee report
  must be attached.  This program must also be approved by the College
  of the student's primary graduate program and by the student's
  dissertation advisor.

\item Of the two departments involved, one must be the student's
  primary affiliation and the other is their secondary affiliation.
  (Their primary dissertation advisor can be in either department.)  The
  degree is then called ``PhD in Primary \& Secondary'' -- for
  example, for a student with a primary affiliation in Chemistry and a
  secondary affiliation in CMSE, the name would be ``PhD in Chemistry
  and Computational Mathematics, Science and Engineering.'' Admission
  requirements to graduate school are based on the primary department.

\item \textbf{Qualifying Exam:} Students whose primary department is
  CMSE must select and pass three of the four subject exams (as
  detailed in Section~\ref{sec:qual_exam}), and students whose
  secondary department is CMSE must select and pass two of the four
  subject exams.  This is typically achieved by the student taking the
  appropriate core CMSE graduate courses and then taking the subject
  exam that is the final exam for that course.  The average of the
  subject exam grades must be at least 3.5, with no one grade being
  less than 3.0, in order for this requirement to be fulfilled. 

\item \textbf{Cognate coursework requirement:}  Students must
  determine, in discussion with their dissertation committee, a
  comprehensive set of courses that fulfills the requirements of both
  departments.  The CMSE PhD program's cognate requirement is
  typically fulfilled by taking coursework in the non-CMSE department,
  with the maximum number of required credits being 120\% of the
  credit requirement in the primary graduate program, excluding
  research credits.  Dual PhD students
  must take a minimum of 12 credits of coursework in
  computationally-focused courses.  This explicitly includes all CMSE
  graduate courses aside from CMSE 801 and 802, and may also include
  computationally-intensive courses in other departments at the
  discretion of the dissertation committee.

\item \textbf{Research credit requirement:}  Students must take at
  least 24, and no more than 36, dissertation research credits in
  their primary department (CMSE 999 or its equivalent).

\item \textbf{Dissertation Committee:} Students must form a PhD dissertation
  committee that includes faculty from both their primary and
  secondary departments, and which satisfies to the greatest extent
  possible the requirements for the composition of a dissertation committee
  from both departments.  The dissertation committee must include at least
  one faculty program advisor whose tenure home is in each of the two
  departments.  The dissertation committee must be formed and meet prior to
  the end of the student's second year in a PhD program in order to
  submit the dual PhD request to the Graduate School.  This meeting
  does not have to be the same committee meeting where the
  comprehensive exam takes place.

\item \textbf{Comprehensive Exam:}  Comprehensive examinations are
  specified according to the guidelines of the primary department, and
  in CMSE the comprehensive examination is generally the first formal
  meeting of the dissertation committee.  This meeting typically
  includes a 
  presentation of the dissertation proposal (although see the previous
  point).  For dual PhDs where CMSE is the secondary department: In
  the case where the comprehensive exam is part of the first formal
  dissertation committee meeting, this meeting should explicitly include
  discussion of the student's career goals and the creation of a
  professional development plan (as detailed in the CMSE PhD program
  description).  In the case where the comprehensive exam takes some
  other form, this discussion should be part of the first formal
  dissertation committee meeting.  This requirement should be fulfilled
  after passage of the qualifying exam, and no later than the end of
 the student's third year.

\item \textbf{Dissertation and dissertation defense:} The student's
  dissertation must be composed of novel research that advances the
  state-of-the-art in algorithms or applications relating to
  computational and/or data science, and must include significant
  intellectual contributions to both disciplines.  The details of the
  dissertation and defense are specified according to the guidelines
  of the primary department.

\item \textbf{Responsible Conduct of Research training:} All PhD
  students must complete Responsible Conduct of Research Training
  through their primary department.

\end{enumerate}

Students whose primary department is CMSE must adhere to the
requirements specified in the CMSE PhD program description
(Section~\ref{sec:phd_requirements}) with regards to the number of
opportunities to pass exams, GPA requirements, and timelines.

If a student decides to leave the interdisciplinary degree program,
their PhD program requirement reverts to the requirements of their
primary affiliation.  In this circumstance, the student should consult
with the graduate director in their primary department to determine if
any further action is needed.
