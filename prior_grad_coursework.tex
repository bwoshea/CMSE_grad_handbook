\section{Policy regarding students entering the CMSE MS or PhD program with prior graduate coursework}
\label{sec:prior_coursework}

\vspace{4mm}
\noindent
\textbf{Applicability of this policy:} This policy was
approved by the CMSE Graduate Studies Committee on May 1, 2017 and is
valid for all CMSE PhD students, regardless of start date (i.e., it
may be applied retroactively to students who started in the PhD
program prior to May 1, 2017).  This policy does not apply to students
pursuing a Master's degree in CMSE (either as a terminal degree or as
a part of their PhD), who must complete a
minimum of 30 credits of graduate-level coursework \textbf{at MSU} to attain the
degree.  Note, however, that this 30-credit limit for the Master's
degree may include credits transferred from
other institutions; see below for more information on this.

\vspace{2mm}
\noindent
\textbf{Reason for this policy:} The CMSE graduate handbook states
that ``Students must take a minimum of 30 credit hours of
non-research-based coursework beyond the Bachelor’s level.''  However,
some students enter the CMSE PhD program with substantial amounts of
prior graduate-level coursework (which may include a M.S., PhD, or
graduate coursework without a degree).  We wish to ensure that
students entering the CMSE PhD program with prior graduate coursework
can have this training taken into account while simultaneously
ensuring that students still receive adequate preparation for their
CMSE dissertation research.

\vspace{2mm}
\noindent
\textbf{Core CMSE courses:} Up to two of the four core CMSE courses
can be waived.  In order for this to happen, the student must provide
a copy of the syllabus for the equivalent course(s) as well as the
name of the textbook(s) used, problem sets/projects, and exams (if
available).  These materials will be shared with the faculty who most
recently taught the equivalent core courses as well as the members of
the Graduate Studies Committee, who will collectively determine if the
courses are approximately equivalent to the CMSE core course and the
student’s achievement is equivalent to at least a 3.5 on the subject
exam.  At the recommendation of the faculty member, that course (and
corresponding subject exam) may be waived.  If the faculty member is
not willing to make such a recommendation, the student has the option
of taking the subject exam at the beginning of the fall semester to
test out of the course and receive credit for the subject exam.  To do
so, they must pass with a minimum grade of 3.5.

\vspace{2mm}
\noindent
\textbf{Cognate coursework:} The purpose of the cognate coursework is
to supplement the core CMSE courses and ensure that students are
well-prepared for their dissertation research.  To this end, the
graduate director, in consultation with the student’s dissertation
advisor and committee, will evaluate the student’s prior graduate
coursework and determine which, if any, of their prior courses relate
to their dissertation research.  Assuming a significant amount of
their prior coursework provides preparation for their dissertation
research, the committee will then determine a coherent set of courses
at MSU that provide broader and/or deeper training for their
dissertation research.  Regardless of prior preparation, students must
take a minimum of 6 credits of coursework beyond the core courses.

\vspace{2mm}
\noindent
\textbf{30 credit requirement (PhD program):} The requirement of a minimum of 30
credits of coursework while enrolled in the CMSE PhD program may be
waived by the Graduate Director, in consultation with the student's
dissertation advisor and dissertation committee.  The overall
guideline for the total amount of coursework is that the student's
relevant prior graduate coursework, plus courses taken while in the
CMSE PhD program, should meet or exceed the equivalent of 30 credits
of graduate-level coursework at MSU. 

\vspace{2mm}
\noindent
\textbf{30 credit requirement (M.S. program):}  The requirement of a
minimum of 30 credits of coursework (including CMSE 899 and/or 891,
depending on whether a student is pursuing a Plan A or Plan B M.S.)
cannot be waived, as it is a university requirement.  This includes
both terminal Master's degrees as well as those pursued during the
course of a PhD in CMSE.  University policy allows for up to 9 credits of graduate-level coursework
to be transferred from another institution, and this transfer credit
can be applied to the M.S. in
CMSE as long as it is in topical areas relevant to the degree
(determined by the CMSE graduate director).  The procedure for doing so is described below.


\vspace{2mm}
\noindent
\textbf{Transferring graduate credits:}  MSU allows up to 9 credits of
graduate
coursework to be transferred from another institution.  A student
wishing to transfer credits must provide an official transcript (which
may already have been done as part of their application to MSU), and
for each course they wish to transfer must provide a course syllabus
including listing of textbooks, representative assignments (homework,
exams, etc.), and a suggestion for the equivalent MSU course.  The
CMSE graduate director, possibly in consultation with the Graduate
Studies Committee, will determine if these courses are at an
appropriate level for transfer as gradouate credits at MSU.  If so,
they will initiate a request for awarding transfer credits to the
student (with the appropriate grade).  In general, courses with a
grade below 3.0 will not be transferred.


\vspace{2mm}
\noindent
\textbf{Extenuating circumstances:} In truly extenuating circumstances
where a student has a particularly large amount of relevant prior
graduate-level training, the Graduate Director, in consultation with
the student’s dissertation advisor and dissertation committee, may
modify the above policy.


