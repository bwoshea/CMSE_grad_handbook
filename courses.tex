\section{Graduate courses}
\label{sec:courses}

\subsection{CMSE graduate courses}

Note: includes cross-listed courses


\noindent\textbf{CMSE 801, Introduction to Computational Modeling.}  Introduction to computational modeling using a wide variety of application examples. Algorithmic thinking and model building, data visualization, numerical methods, all implemented as programs. Command line interfaces. Scientific software development techniques including modular programming, testing, and version control.  Recommended background: one semester of introductory calculus.  \textbf{(3 credits)}  
\vspace{3mm}

\noindent\textbf{CMSE 802, Methods in Computational Modeling.}  Standard computational modeling methods and tools. Programming and code-management techniques.  Recommended background:  CMSE 801 or equivalent experience.  \textbf{(3 credits)}  

\vspace{3mm}
\noindent\textbf{CMSE 820, Mathematical Foundations of Data Science.}  Introduces students to the fundamental mathematical principles of data science that underlie the algorithms, processes, methods, and data-centric thinking. Introduces students to algorithms and tools based on these principles.  Recommended background:  CMSE 802 or equivalent experience.  Differential equations at the level of MTH 235/255H/340+442/347H+442.  Linear algebra at the level of MTH 390/317H.  Probability and statistics at the level of STT 231.  \textbf{(3 credits)}

\vspace{3mm}
\noindent\textbf{CMSE 821, Numerical Methods for Differential Equations.}
Numerical solution of ordinary and partial differential equations, including hyperbolic, parabolic, and elliptic equations. Explicit and implicit solutions. Numerical stability.
Recommended background:  CMSE 802 or equivalent experience.  Differential equations at the level of MTH 235/255H/340+442/347H+442.  Linear algebra at the level of MTH 390/317H.  \textbf{(3 credits)}

\vspace{3mm}
\noindent\textbf{CMSE/CSE 822, Parallel Computing.}  Core principles and techniques of parallel computation using modern supercomputers. Parallel architectures. Parallel programming models. Principles of parallel algorithm design. Performance analysis and optimization. Use of parallel computers.  Recommended background: One semester of introductory calculus. Ability to program proficiently in C/C++, basic understanding of data structures and algorithms (both at the level of CSE 232). Basic linear algebra and differential equations.  \textbf{(3 credits)}

\vspace{3mm}
\noindent\textbf{CMSE 823, Numerical Linear Algebra, I.}  Convergence and error analysis of numerical methods in applied mathematics.  Recommended background: CMSE 802 or equivalent experience; Linear algebra at the level of MTH 414. \textbf{(3 credits)}

\vspace{3mm}
\noindent\textbf{CMSE 890, Selected Topics in Computational Mathematics, Science, and Engineering.}  Topics selected to supplement and enrich existing courses and lead to the development of new courses.  Recommended background varies with topic and instructor.  \textbf{(1-4 credits)}  Note: A student may earn a maximum of 12 credits in all enrollments fog this course.

\vspace{3mm}
\noindent\textbf{CMSE 891, Independent Study in Computational Mathematics, Science, and Engineering.}  \red{need to fill this in.}

\vspace{3mm}
\noindent\textbf{CMSE 899, Master's Thesis Research.}  Master's thesis research.  \textbf{(1-6 credits)}  Note: A student may earn a maximum of 8 credits in all enrollments for this course.

\vspace{3mm}
\noindent\textbf{CMSE 999, Doctoral Dissertation Research.}  Doctoral dissertation research.  \textbf{(1-24 credits)}   Note: A student may earn a maximum of 36 credits in all enrollments for this course.

\vspace{3mm}

\subsection{Non-CMSE computational and data-science courses}

\textbf{Note:} this list contains courses that have been pre-screened and will automatically be accepted for the CMSE graduate certificates and degrees (modulo limits described in the individual program descriptions).  Please note that other computationally-focused MSU courses may also be acceptable for these programs!  Consult departmental enumerateings in the [MSU course catalog](http://reg.msu.edu/Courses) for the most timely information about appropriate courses, and email the [CMSE Director of Graduate Studies](mailto:fill@this.in) if you have questions about courses that may count toward a CMSE graduate certificate or degree.

\subsubsection{Courses at the 400 level}

\begin{itemize}
\item BMB/MMG/PLB-400, Introduction to Bioinformatics (3 credits)  
\item CEM-481, Computational chemistry (3 credits)  
\item ME-475, The Use of Finite Element Methods (3 credits)  
\item MTH-451, Numerical Analysis, I (3 credits)  
\item MTH-452, Numerical Analysis, II (3 credits)  
\item PHY-480, Computational Physics (3 credits)  
\item STT-461, Computations in Probability and Statistics (3 credits)  
\item STT-465, Bayesian Statistical Methods (3 credits)  
\end{itemize}

\subsubsection{Courses at the 800 and 900 level}

\begin{itemize}
\item AST-911, Numerical techniques in astronomy (3 credits)  
\item CE-822, Ground water modeling (3 credits)  
\item CE-823, Stochastic ground water modeling (3 credits)  
\item CE/ME-872, Finite element methods (3 credits)  
\item CEM-883, Computational quantum chemistry (3 credits)  
\item CEM-888, Computational chemistry (3 credits)  
\item CSE-836, Prob. Models and Algorithms in Comp. Bio. (3 credits)  
\item CSE-845, Multi-disc. rsrch. meth. for study of evolution (3 credits)  
\item CSE-881, Data Mining (3 credits)  
\item CSE-912, Artificial life communities in science and engineering (3 credits)  
\item ECE-837, Comp. methods in electromagnetics (3 credits)  
\item ME-835, Turbulence modeling and simulation (3 credits)  
\item ME-840	Comp. fluid dynamics and heat transfer (3 credits)  
\item MTH-850, Numerical Analysis, I (3 credits)  
\item MTH-851, Numerical Analysis, II (3 credits)  
\item MTH-852, Numerical Methods for ODEs (3 credits)  
\item MTH-950, Numerical Methods for PDEs (3 credits)  
\item MTH-951, Numerical Methods for PDEs, II (3 credits)  
\item MTH-995, Special Topics in Numerical Analysis (3 or more credits)  
\item PHY-915, Computational Condensed Matter Physics (2 credits)  
\item PHY-919, Modern Electronic Structure Theory (2 credits)  
\item PHY-950, Data analysis methods (2 credits)  
\item PHY-998, Computational Tools for Nuclear Physics (2 credits)  
\item PLB-810, Theories and practices in bioinformatics (3 credits)  
\item PSY-992, Computer programming for behavioral scientists (3 credits)    
\item QB-826, Intro to Quantitative Biology Techniques (1 credit)  
\item STT-802, Statistical Computation (3 credits)  
\item STT-874, Introduction to Bayesian Analysis (3 credits) 
\end{itemize}