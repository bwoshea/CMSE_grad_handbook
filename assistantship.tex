\section[Policies relating to graduate assistantships]{Policies
relating to graduate assistantships\footnote{Adapted with permission
from the MSU Department of Physics \& Astronomy
\href{https://www.pa.msu.edu/grad/GradHandbook_Aug2015.pdf}{Handbook
for Graduate Students}, Section XIV}}

All doctoral students in the Computational
Mathematics, Science and Engineering PhD program are expected to be
employed as either assistants in our teaching programs (TA) or as
assistants in one of the research groups within the department (RA).
Professional behavior is expected from students in these positions, and
students in our program carry out their duties at a high level of
performance.  Teaching Assistants are governed by the MSU/GEU contract
(\url{http://grad.msu.edu/geu/}).  Incoming students may be supported
by either a TA or an RA; after their first two years, all students are
expected to be supported by research assistantships for the duration
of their PhD program.

\subsection{Teaching Assistantships}

Graduate Students who are assigned teaching assistantships are
\textbf{required to attend} a TA Orientation session prior to their first
semester as a TA at Michigan State University.  These sessions may be
offered by the College of Engineering or the Graduate School.  You
will receive an email from the Graduate Director with instructions in
this regard, if relevant.

International students who are not native speakers of English must
take the \textbf{SPEAK} test and pass the examination at the required
level in order to be appointed as a TA.  Students must have a score of
at least 50 or waiver approval following an interview to satisfy the
SPEAK test requirement.

International students with a teaching assistantship are
\textbf{required to attend} the International Teaching Assistant
Orientation offered by the Graduate School.

Decisions on TA appointments are made by the Director of Graduate
Studies.  Students will be informed by the end of March whether they
will have a TA position for the following academic year, subject to
continued satisfactory progress in their Ph.D. program, subject to
continued adequate performance of their TA duties, and subject to the
budgetary considerations.

Important factors in making these decisions are:

\begin{itemize}
\item Progress through the core PhD program coursework and 
cognate courses.
\item Professional and courteous performance of TA duties.
\item Identifying research opportunities and making adequate progress
towards their degree.
\end{itemize}

\subsection{Research Assistantships}

Decisions on RA appointments are made by individual faculty or by
faculty groups involved in group research projects. Students will be
informed by the end of March whether they will receive an RA for the
following academic year (or whether their RA will be continued for
the following academic year), subject to satisfactory performance of
their RA duties and subject to the budgetary considerations.

Students should seek an advisor with a RA opening before the end of
their second year in the program.

\subsection{Missed work}

In this section, ``missed work'' refers to work missed due to any
reason, including (but not limited to) illness, injury, pregnancy, or
dereliction of duty.  Policies regarding missed work are governed by
Article 12. iv-v and Article 18 of the
\href{https://www.hr.msu.edu/documents/contracts/GEU2015-2019.pdf}{GEU
contract} and, for students not covered by the GEU contract, the
University's
\href{https://reg.msu.edu/AcademicPrograms/Text.aspx?Section=111#s351}{GA
Illness, Injury, and Pregnancy Leave policy}.  Note that missed work
due to illness, injury, or pregnancy are subject to a very different
set of policies than missed work due to unsatisfactory performance --
please consult the documents listed above for more information.


\subsection{Outside work for pay}

Graduate students who are appointed as a TA
or an RA are expected to devote their time to their academic studies
and to their TA/RA responsibilities. No outside work for pay can be
undertaken without discussing with Director of Graduate Studies (in
the case of TAs) or with their research advisors (in the case of RAs).

\subsection{Tutoring}

Tutoring can be of tremendous benefit to you as a student, both
intellectually and financially.  It can help you gain a better
understanding of your field, and also help you to improve your
teaching skills.  It is critical, however, that tutoring not interfere
with your coursework and with your assistantship.  As such, you should
discuss the decision to tutor with your academic and/or research
advisor.  Furthermore, tutoring should not exceed an average of 5
hours of your time per week.

If you are a teaching assistant, you \textbf{may not} receive
compensation to tutor students enrolled in the course you are assigned
to.  Helping students in all sections of the course counts as a part
of your duties, whether it occurs through office hours, direct contact
in class, or in less formal settings.  Asking for pay would constitute
a conflict of interest because you are already being paid by the
Department to provide these services for that particular course.  You
may, however, act as a paid tutor for any course to which you are
\textbf{not} assigned as a TA in any given semester once you have
received permission from your academic and/or research advisor.
