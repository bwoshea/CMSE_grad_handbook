\section{Policies relating to graduate assistantships}\footnote{Physics grad
  handbook XIV}

Most Ph.D. students are employed by the Physics and Astronomy
department, either as assistants in our teaching programs (TA) or as
assistants in one of the research groups within the department (RA).
Professional behavior is expected from students in these positions and
students in our program carry out their duties at a high level of
performance. The TA's are governed by the MSU/GEU contract
(\url{http://grad.msu.edu/geu/}).  Incoming students may be supported
by either a TA or an RA; after their first two yeras, all students are
expected to be supported by research assistantships for the duration
of their PhD program.

International students who are not native speakers of English must
take the \textbf{SPEAK} test and pass the examination at the required
level in order to be appointed as a TA. Students must have a score of
at least 50 or waiver approval following an interview to satisfy the
SPEAK test requirement.

Decisions on TA appointments are made by the director of graduate
studies.  Students will be informed by the end of March whether they
will have a TA position for the following academic year, subject to
continued progress in their Ph.D. program, subject to continued
adequate performance of their TA duties and subject to the budgetary
considerations.

Important factors in making these decisions are:

\begin{itemize}
\item Progress through the core PhD program coursework and their
cognate course requirement.
\item Professional and courteous performance of TA duties.
\item Identifying research opportunities and making adequate progress
towards their degree.
\end{itemize}

Decisions on RA appointments are made by individual faculty or by
faculty groups involved in group research projects. Students will be
informed by the end of March whether their RA will be continued for
the following academic year, subject to satisfactory performance of
their RA duties and subject to the budgetary considerations.

Students should seek an advisor with a RA opening before the end of
their second year in the program.

\textbf{Missed work:} If students are sick or otherwise unable to
complete their TA or RA duties, they must inform their TA or RA
advisor immediately. Students who fail to carry out their duties and
who fail to give an adequate reason for their absence will be sent a
warning letter immediately. If the student fails to respond
appropriately, the student’s stipend will be stopped 10 days after the
warning letter is sent.

\textbf{Outside work for pay:} The students who are appointed as a TA
or a RA are expected to devote their time to their academic studies
and to their TA/RA responsibilities. No outside work for pay can be
undertaken without discussing with Director of Graduate Studies (in
the case of TAs) or with their research advisors (in the case of RAs).

\textbf{Tutoring:} Tutoring can be of tremendous benefit to you as a
student, both intellectually and financially.  It can help you gain a
better understanding of your field, and also help you to improve your
teaching skills.  It is critical, however, that tutoring not interfere
with your coursework and with your assistantship.  As such, you should
discuss the decision to tutor with your advisor.  Furthermore,
tutoring should not exceed an average of 5 hours of your time per
week.

If you are a teaching assistant, you may not receive compensation to
tutur students enrolled in the course you are assigned to.  Helping
students in all sections of the course counts as a part of your
duties, whether it occurs through office hours, direct contact in
class, or in less formal settings.  Asking for pay would constitute a
conflict of interest because you are already being paid by the
Department to provide these services for that particular course.  You
may, however, act as a paid tutor for any course to which you are
\textbf{not} assigned as a TA in any given semester.
