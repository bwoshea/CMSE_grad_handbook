\section[Policy regarding appeals]{Policy regarding
  appeals}
\label{sec:appeals}

Students who have failed either the CMSE Qualifying exam or the
Comprehensive exam, and believe that an appeal of this result is
warranted, may appeal the outcome.  In the context of the CMSE
Qualifying exam, failure may include receiving a grade of 2.5 or below
on one or more subject exams, or failing to achieve the average
subject exam grade requirement described in
Section~\ref{sec:qual_exam}, or both of these circumstances.  The
student may appeal only after the maximum number of opportunities to
take the subject exams has occurred or the time limit for passage has
elapsed, whichever comes first.  In the context of the CMSE
Comprehensive exam, students may appeal only after receiving a grade
of ``Fail'' from the dissertation committee twice in a row, or after
having failed to accomplish the remediation required by the
dissertation committee in a reasonable time after receiving a ``Pass
with qualification'' (with both of these terms being described in
Section~\ref{sec:comp_exam}).

Appeals must be made in writing and sent to to the Department's
Graduate Director within two weeks of the date that the student is
notified of failure of the Qualifying or Comprehensive examinnation.
The written appeal must contain (1) explicit reasons for requesting
that the review be conducted, (2) an explanation of how the student
will remediate the deficiencies indicated by failure of the exam(s) in
question, and (3) a request for a specific outcome of the appeal.
Students are strongly advised to consult with their advisor and/or the
Graduate Director before submitting this appeal.

After receipt of the appeal, the Graduate Director will ask the
student's dissertation advisor (or current research supervisor, if a
dissertation advisor has not yet been identified) to provide written
feedback on the student's performance in the graduate program up to
that time, including coursework and research, and will also be asked
to give an evaluation of the likelihood that the student will be able
to successfully complete the doctoral program with an acceptable level
of research productivity and in a reasonable amount of time.  The
graduate director may also ask for further written information from
the faculty who wrote and/or graded any subject exams that are
relevant to the appeal, or from members of the student's dissertation
committee regarding the Comprehensive exam.  The CMSE Graduate Studies
committee will meet to discuss the student's appeal and vote on it at
the next GSC meeting following the receipt of the appeal and
supplemental information, with the graduate student member of the
committee not participating in the discussion and vote.  In addition
to the student’s appeal and written information from the advisor
and/or other faculty, other information about the student may be
considered (e.g., transcripts, their Grad Plan, etc.).

The Graduate Director will inform the student and their advisor of the
decision of the Graduate Studies Committee in writing as soon as
possible after the GSC meeting where the vote took place.  This letter
will also discuss either steps that must be taken as part of the
successful appeal, or the student’s possible next action(s) if the
appeal is unsuccessful.

Note that regardless of other circumstances, petition requests will
only be granted for students whose level of research progress, as
reported by their dissertation advisor, makes it clear that they are
highly likely to successfully complete the doctoral program with an
acceptable level of research productivity and in a reasonable amount
of time.
